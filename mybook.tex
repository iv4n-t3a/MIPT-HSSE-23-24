\documentclass[a6paper, 10pt]{book}

\usepackage[english, russian]{babel}
\usepackage[T2A]{fontenc}
\usepackage[utf8]{inputenc}
\usepackage{amsmath, amsfonts, amssymb, amsthm, mathtools}
\usepackage{indentfirst}
\usepackage{soulutf8}
\usepackage{geometry}
\theoremstyle{plain}

\newtheorem{theorem}{Теорема}
\newtheorem*{theorem_}{Теорема}
\newtheorem{corollary}{Следствие}[theorem]
\newtheorem{corollary_}{Следствие}[theorem_]

\geometry{top=20mm}
\geometry{bottom=20mm}
\geometry{left=10mm}
\geometry{right=10mm}

\newcommand{\N}{\mathbb N}
\newcommand{\Z}{\mathbb Z}
\newcommand{\Q}{\mathbb Q}
\newcommand{\R}{\mathbb R}
\newcommand{\eps}{\varepsilon}
\renewcommand{\phi}{\varphi}
\renewcommand{\kappa}{\varkappa}
\newcommand{\larrow}{\Leftarrow}
\newcommand{\rarrow}{\Rightarrow}
\newcommand{\lrarrow}{\Leftrightightarrow}


\begin{document}
	\pagenumbering{gobble}
	\author{Иван-Чай}
	\date{1984}
	\title{Математический анализ рептилоидов}

	\linespread{1.4}
	\selectfont

	\maketitle
	\newpage

	\pagenumbering{arabic}

	\begin{theorem_}
		\[ % \[
		\left[
		\begin{gathered}
			\alpha \neq 8
			\\
			\alpha \neq 13
			\\
		\end{gathered}
		\forall \alpha : \alpha \in \N, \alpha \geq 5
		\] % \[
	\end{theorem_}

	\begin{proof}
		Рассмотрим случаи

		\begin{itemize}
			\item $ \alpha > 11 \rarrow \alpha \neq 8 $
			\item $ \alpha \leq 11 \rarrow \alpha \neq 13 $
		\end{itemize}
	\end{proof}

	Теперь рассмотрим следущую совокупность уравнений
	\[ % \[
		\left[
		\begin{gathered}
			(x - 1.4)^2 + y^2 = 1
			\\
			(x + 1.4)^2 + y^2 = 1
			\\
			x^2 + \left(\frac{y}{4} - 2\right)^2 = 1
		\end{gathered}
	\] % \]

	Она имеет график следущего вида

	\includegraphics[width=1 \textwidth]{~/Pictures/hui.png}

	Свойства этой совокупности предлагаются читателю для самостоятельного изучения, в качестве несложного упражнения
\end{document}

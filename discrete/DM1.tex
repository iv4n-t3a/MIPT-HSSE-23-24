\documentclass[a6paper, 10pt]{article}

\usepackage[english, russian]{babel}
\usepackage[T2A]{fontenc}
\usepackage[utf8]{inputenc}
\usepackage{amsmath, amsfonts, amssymb, amsthm, mathtools}
\usepackage{indentfirst}
\usepackage{soulutf8}
\usepackage{geometry}
\usepackage{ulem}
\theoremstyle{plain}

\newtheorem*{theorem}{Th}
\newtheorem{theorem_}{Th}
\newtheorem*{statement}{St}
\newtheorem{statement_}{St}
\newtheorem{corollary}{Cl}[theorem]
\newtheorem{corollary_}{Cl}[theorem_]

\geometry{top=20mm}
\geometry{bottom=20mm}
\geometry{left=10mm}
\geometry{right=10mm}

\newcommand{\N}{\mathbb N}
\newcommand{\Z}{\mathbb Z}
\newcommand{\Q}{\mathbb Q}
\newcommand{\R}{\mathbb R}
\newcommand{\eps}{\varepsilon}
\renewcommand{\phi}{\varphi}
\renewcommand{\kappa}{\varkappa}
\newcommand{\larrow}{\Leftarrow}
\newcommand{\rarrow}{\Rightarrow}
\newcommand{\lrarrow}{\Leftrightarrow}


\begin{document}
	\pagenumbering{gobble}
	\author{Зухба Анастасия Викторовна\\(Конспектировал Иван-Чай)}
	\date{01.09.2023}
	\title{Дискретная математика}

	\linespread{1.4}
	\selectfont

	\maketitle
	\newpage

	\section{Основные основы}

	Граф - это пара множеств
	$ G(V, E) $
	$ V $ - множество, элементы которого называют вершинами
	$ E $ - множество пар вершин $ (U, U'): U, U' \in V $

	Вершина и ребро инцидентны, если вершина является одним из концов ребра.

	\sout{Петлей называется ребро вида $ (U, U) $.}

	\sout{Кратными называются ребра соответствующие одной и той же паре $ (U, U') $.}

	Петли и кратные ребра в нашем курсе использоваться не будут.

	Степенью вершины $ U $неориентированного гарафа $ G(V, E) $ называется
	количество ребер инцидентных $ U $.

	\[ % \[
		\deg{U} = \left| \left\{ (U, U'): (U, U') \in E \right\} \right|
	\] % \]

	Смежные вершины - это вершины с общим ребром.

	Смежные ребра - это ребра с общей вершиной.

	\begin{theorem}
	Лемма о рукопожатиях:

	\[ % \[
	\sum_{u \in V} \deg{U} = 2 \left| E \right|
	\] % \]
	\end{theorem}

	\section{Хранение графа в памяти}

	\subsection{Матрица смежности}

	Есть матрица $ A $ размером $ \left| V \right| * \left| V \right| $, и
	$ a_{ij} = 1 $, если $ (i, j) \in E $, иначе $ a_{ij} = 0 $.

	\subsection{Матрица инцидентности}

	Есть матрица $ A $ размера $ \left| V \right| * \left| E \right| $
	и там каждый столбец описывает ребро тип.

	\subsection{Список смежности}

	ну там и так все знают, как и предыдущие два пункта.

	\section{Маршруты, пути, цепи, связность}

	Маршрут - конечная последовательность вершин такая, что
	$ (U_j, U_{j+1}) \in E \quad \forall j \in \overline{1, \left|V\right|-1} $.

	Цепь - маршрут, все ребра которого различны.

	Путь в нашем курсе - синоним цепи.

	Простой путь - путь, все вершины которого (кроме, возможно, начала и конца) различны.

	Замкнутый маршрут/путь/простой путь, это когда совпадают начало и конец.

	Цикл - замкнутый путь.

	Длиной маршрута называется число ребер в нем с учетом возможных повторных вхождений.

	Вершины $ U $ и $ V $ называют связными, если существует маршрут
	с началом в $ U $ и концом в $ U $.

	Граф называют связным, если между любыми двумя его вершинами существует путь.

	\begin{statement}
	\exists $ маршрут из $ U $ в $ V $ \rarrow  \exists $ путь и простой путь из $ U $ в $ V $.
	\end{statement}

	\begin{proof}
		Пусть $ \left( U_1, U_2, U_3 \dots U_m \right) $ - маршрут.

		Предположим все вершины различны, тогда этот маршрут является простым путем.

		Предположим маршрут содержит повторение вершины $ U_i $


		Пусть $ U_i = U_j $ и

	\begin{allign}
		\left( U_1, U_2, U_3 \dots U_{i-2}, U_{i-1} \right) = a

		\left( U_{i+1}, U_{i+2}, U_{i+3} \dots U_{j-2}, U_{j-1} \right) = b

		\left( U_{j+1}, U_{j+2}, U_{j+3} \dots U_{k-1}, U_{k} \right) = c
	\end{allign}

	тогда $ a U_i b U_j c $ - исходный маршрут, но тогда $ a U_i c $ - тоже маршрут.

	\end{proof}
\end{document}

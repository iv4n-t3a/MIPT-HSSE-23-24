% /*
%  * ----------------------------------------------------------------------------
%  * "THE BEER-WARE LICENSE" (Revision 42):
%  * @iv4n-t3a wrote this file.  As long as you retain this notice you
%  * can do whatever you want with this stuff. If we meet some day, and you think
%  * this stuff is worth it, you can buy me a beer in return.   Иван-Чай
%  * ----------------------------------------------------------------------------
%  */

\documentclass[a5paper, 10pt]{article}

\usepackage[english, russian]{babel}
\usepackage[T2A]{fontenc}
\usepackage[utf8]{inputenc}
\usepackage{amsmath, amsfonts, amssymb, amsthm, mathtools}
\usepackage{indentfirst}
\usepackage{soulutf8}
\usepackage{geometry}
\usepackage{ulem}
\usepackage{color}
\usepackage{hyperref}

\theoremstyle{plain}

\newtheorem{theorem}{Th}
\newtheorem*{theorem_}{Th}
\newtheorem*{statement}{St}
\newtheorem{statement_}{St}
\newtheorem{definition}{Def}
\newtheorem*{definition_}{Def}
\newtheorem{lemma}{Lem}
\newtheorem*{lemma_}{Lem}
\newtheorem*{note}{Nt}
\newtheorem{exersise}{Ex}
\newtheorem{corollary}{Cl}[theorem]
\newtheorem{corollary_}{Cl}[theorem_]

\geometry{top=20mm}
\geometry{bottom=20mm}
\geometry{left=10mm}
\geometry{right=10mm}

\newcommand{\N}{\mathbb N}
\newcommand{\Z}{\mathbb Z}
\newcommand{\Q}{\mathbb Q}
\newcommand{\R}{\mathbb R}
\newcommand{\eps}{\varepsilon}
\renewcommand{\phi}{\varphi}
\renewcommand{\kappa}{\varkappa}
\renewcommand{\vec}{\overrightarrow}

\newcommand{\oR}{\overline{\mathbb R}}
\newcommand{\hR}{\hat{\mathbb R}}

\newcommand{\larrow}{\leftarrow}
\newcommand{\rarrow}{\rightarrow}
\newcommand{\lrarrow}{\leftrightarrow}
\newcommand{\hrarrow}{\hookrightarrow}
\newcommand{\Larrow}{\Leftarrow}
\newcommand{\Rarrow}{\Rightarrow}
\newcommand{\Lrarrow}{\Leftrightarrow}

\hypersetup{
	linktocpage,
    colorlinks=true,
    linktoc=all,
    linkcolor=blue,
}


\begin{document}
	\pagenumbering{gobble}
	\author{Зухба А.В.\\(Конспектировал Иван-Чай)}
	\date{6 лекция}
	\title{Дискретна математика}

	\linespread{1.4}
	\selectfont

	\maketitle
	\newpage

	\tableofcontents

    \section{Правило включений-исключений}

    \[
        | A \cup B | = | A | + | B | - | A \cap B |
    .\] \[
        | A \cup B \cup C | = | A | + | B | + | C | -
        | A \cap B | - | A \cap C | - | B \cap C | +
        | A \cap B \cap C |

    .\]

    Пусть
        $ x \in X $,
        $ \alpha_1, \alpha_2, \alpha_3, \dots \alpha_n $,
        $ N(\alpha_{j_1}, \alpha_{j_2}, \dots \alpha_{j_k}) $ -
        количество жлементов обладающих свойсвтами
        $ \alpha_{j_1}, \alpha_{j_2}, \dots \alpha_{j_k} $.
    \[
        N_0 = N(\neg \alpha_1, \neg \alpha_2, \dots \neg \alpha_n)
    .\] \[
        N_0 = N - N(\alpha_1) - N(\alpha_2) + N(\alpha_1, \alpha_2)
    .\]

    В общем виде:
    \[
        N_0 = N - \sum\limits_j N(\alpha_j) +
        \sum \limits_{1 \leq \alpha_{j_1} < \alpha_{j_2} \leq n}
        N(\alpha_{j_1}, \alpha_{j_2}) - \dots
        + (-1)^n N(\alpha_1, \alpha_2, \dots \alpha_n) (*)
    .\]
    Слогаемое связаные с k свойствами:
    \[
        (-1)^k \sum\limits_{1 \leq \alpha_{j_1} < \dots < \alpha_{j_k} \leq n}
        (\alpha_{j_1}, \alpha_{j_2}, \dots \alpha_{j_k})
    .\]

    Докажем во общем виде

    Пусть (*) выполняется для любого набора из не более чем n в свойств
    \[
        N(
            \neg \alpha_1,
            \neg \alpha_2,
            \dots
            \neg \alpha_n,
            \neg \alpha_{n + 1}
        ) = N(
            \neg \alpha_1,
            \neg \alpha_2,
            \dots
            \neg \alpha_n,
        ) - N(
            \neg \alpha_2,
            \neg \alpha_3,
            \dots
            \neg \alpha_{n + 1}
        )
    .\]
    Распишем
    для элементов обладающих свойсвтом $ \alpha_{n + 1} $
    \[
        N(
            \neg \alpha_1,
            \neg \alpha_2,
            \dots
            \neg \alpha_n,
            \neg \alpha_{n + 1}
        ) =
    \] \[
        = N(\alpha_{n + 1}) -
            \sum\limits_{j = 1}^n N(\alpha_j, \alpha_{n+1}) + \dots +
        (-1)^k \sum\limits_{1 \leq \alpha_{j_1} < \dots < \alpha_{j_k} \leq n}
        (\alpha_{j_1}, \alpha_{j_2}, \dots \alpha_{j_k}, \alpha_{n+1}) =
    \] \[
        = N(\alpha_{n + 1}) -
            \sum\limits_{j = 1}^n N(\alpha_j, \alpha_{n+1}) + \dots +
        (-1)^k \sum\limits_{1 \leq j_1 < \dots < j_{k-1} < j_k \leq n}
        (\alpha_{j_1}, \alpha_{j_2}, \dots \alpha_{j_k}, \alpha_{n+1}) =
    \] \[
        = N(\alpha_{n + 1}) -
            \sum\limits_{j = 1}^n N(\alpha_j, \alpha_{n+1}) + \dots +
        (-1)^k \sum\limits_{1 \leq j_1 < \dots < j_{k-1} < j_k < n + 1 }
        (\alpha_{j_1}, \alpha_{j_2}, \dots \alpha_{j_k}, \alpha_{n+1})
    \]

    \section{Инъективное биективное и сюръективное отображение    }

    \begin{definition}
        Задано отображение или функция \gamma: X \to Y, если
        каждому элементу $ x \in X $ сопоставлен единственный элемент
        $ \gamma(x) \in Y $.
    \end{definition}

    \begin{definition}
        Im \gamma = \gamma(x) = \{ y \in Y | x \in X: \gamma(x) \in Y \}.
    \end{definition}

    \begin{definition}
        \gamma^{-1}(y) = \{ x \in X | f(x) = y \} \subseteq X.
    \end{definition}

    \begin{definition}
        Сюрьективное отображение (из X \bold{на} Y) Im \gamma = Y.
    \end{definition}

    Нет пустых ящиков:)

    \begin{definition}
        инъективное отображение $ x \neq y \Rarrow \gamma(x) \neq \gamma(y) $.
    \end{definition}

    В ящиках не более одного шарика:)

    \begin{definition}
        Биекция = сюръекция + инъекция.
    \end{definition}

    В каждом ящике по 1 шарику:)

    \section{Какая-то табличка}

    р - различимо, нр - неразличимо.

    \noindent
    \begin{tabular}{|h| a | b | c | d |}
        \hline
         & произвольно & инъект.($ m \geq n $) & сюръет. ($ n \geq m $) & биект.(n = m) \\ \hline
        X, Y - р        & $ m^n $ & $ \frac{m!}{(m - n)!} & ? & $ m! $ \\ \hline
        X - нр, Y - р   & C_{n+m-1}^n & C_n^m & $ C_{n-1}^{m-1} & 1 \\ \hline
        X - р,  Y - нр  & ? & 1 & ? & 1 \\ \hline
        X, Y - нр       & ? & 1 & ? & 1 \\ \hline
    \end{tabular}
\end{document}

% /*
%  * ----------------------------------------------------------------------------
%  * "THE BEER-WARE LICENSE" (Revision 42):
%  * @iv4n-t3a wrote this file.  As long as you retain this notice you
%  * can do whatever you want with this stuff. If we meet some day, and you think
%  * this stuff is worth it, you can buy me a beer in return.   Иван-Чай
%  * ----------------------------------------------------------------------------
%  */

\documentclass[a5paper, 10pt]{article}

\usepackage[english, russian]{babel}
\usepackage[T2A]{fontenc}
\usepackage[utf8]{inputenc}
\usepackage{amsmath, amsfonts, amssymb, amsthm, mathtools}
\usepackage{indentfirst}
\usepackage{soulutf8}
\usepackage{geometry}
\usepackage{ulem}
\usepackage{color}
\usepackage{hyperref}

\theoremstyle{plain}

\newtheorem{theorem}{Th}
\newtheorem*{theorem_}{Th}
\newtheorem*{statement}{St}
\newtheorem{statement_}{St}
\newtheorem{definition}{Def}
\newtheorem*{definition_}{Def}
\newtheorem{lemma}{Lem}
\newtheorem*{lemma_}{Lem}
\newtheorem*{note}{Nt}
\newtheorem{exersise}{Ex}
\newtheorem{corollary}{Cl}[theorem]
\newtheorem{corollary_}{Cl}[theorem_]

\geometry{top=20mm}
\geometry{bottom=20mm}
\geometry{left=10mm}
\geometry{right=10mm}

\newcommand{\N}{\mathbb N}
\newcommand{\Z}{\mathbb Z}
\newcommand{\Q}{\mathbb Q}
\newcommand{\R}{\mathbb R}
\newcommand{\eps}{\varepsilon}
\renewcommand{\phi}{\varphi}
\renewcommand{\kappa}{\varkappa}

\newcommand{\oR}{\overline{\mathbb R}}
\newcommand{\hR}{\hat{\mathbb R}}

\newcommand{\larrow}{\leftarrow}
\newcommand{\rarrow}{\rightarrow}
\newcommand{\lrarrow}{\leftrightarrow}
\newcommand{\hrarrow}{\hookrightarrow}
\newcommand{\Larrow}{\Leftarrow}
\newcommand{\Rarrow}{\Rightarrow}
\newcommand{\Lrarrow}{\Leftrightarrow}

\hypersetup{
	linktocpage,
    colorlinks=true,
    linktoc=all,
    linkcolor=blue,
}


\begin{document}
	\pagenumbering{gobble}
	\author{Зухба Анастасия Викторовна\\(Конспектировал Иван-Чай)}
	\date{5 лекция}
	\title{Дискретная математика}

	\linespread{1.4}
	\selectfont

	\maketitle
	\newpage

	\tableofcontents

    \section{Сочетания}

    $ C_n^k $ или $ \big(^n_k\big) $ - сочетания ( выбор без учета порядка)
    $ k $ элементов из $ n $ различных.

    \[
        (1 + x)^n = (1 + x_1 )(1 + x_2 )(1 + x_3 ) \dots = + x_{j_1 } x_{j_2 } x_{j_3 } x_{j_4} \dots

    .\]

    Треугольник Паскаля
    % Не бейти
    \[
        1 \quad 1
    \]\[
        1 \quad 2 \quad 1
    \]\[
        1 \quad 3 \quad 3 \quad 1
    .\]
    \[
        C^0_1 \quad C^1_1
    \]\[
        C^0_2 \quad C^1_2 \quad C^2_2
    \]\[
        C^0_3 \quad C^1_3 \quad C^2_3 \quad C^3_3
    .\]

    \begin{statement}
        C_{t+1}^k = C_t^{k - 1} + C_t^k.
    \end{statement}

    \begin{proof}[Комбинаторное доказательство]
        Выделим некий элемент и разделим все сочетания из t + 1 по k на содержащие его и нет.
    \end{proof}

    \begin{proof}[Алгеброическое доказательство]
        \[ \frac{(t+1)!}{k!(t + 1 - k)!} =  \frac{1}{(k-1)! (t - k + 1)!} +  \frac{t!}{k!(t-k)!} \]
    \end{proof}

    \begin{proof}[Среднее доказательство]
        \[
            (1 - x )^{t + 1} = (x + 1)(x + 1)^t
        .\]
    \end{proof}

    \begin{proof}[Еще одно доказательство]
    Повернем треугольник паскаля

    \noindent
    $ 1 $ \newline
    $ 1 \quad 4 $ \newline
    $ 1 \quad 3 \quad 6 $ \newline
    $ 1 \quad 2 \quad 3 \quad 4 $ \newline
    $ 1 \quad 1 \quad 1 \quad 1 \quad 1 $ \newline

    Значение в каждой клетке равно количеству способов прийти в нее из нижнего левого угла
    и $ C^n_{n + m} $
    \end{proof}

    \section{Сочетания с повторениями}

    $ \overline{C_n^k} $ - сочетания (без учета порядка) с повторениями из $ n $ элементного множества.
    $ \overline{C_n^k} = C _{n - 1 + k}^k $ - Доказательство в предыдущей лекции.

    Решить в целых числах уравнение
    \[
        x_1 + x_2 + x_3 \dots + x_n = k
    .\]
    Количество решений $ \overline{ C_n^k } $

    \[
        (1 + x)^n =
            C_n^0 x^0 +
            C_n^1 x^1 +
            \dots
            C_n^n x^n
    .\]
    Продиффиренцируем
    \[
        n(1 + x)^{n-1} =
            0 +
            1 C_n^1 x^0 +
            2 C_n^2 x^1 +
            \dots
            (k - 1) C_n^k x^{k - 1}
            \dots
            (n - 1) C_n^n x^{n - 1}
    .\]
    \[
        n(1 + x)^{n-1} =
            \sum_{k = 0}^{n} k C_n^k x^{k-1} \quad (*)
    \]

     \begin{statement}
        \[
            \sum_{k = 0}^{n} C_n^k = 2^n \quad (*, x = 1)
        .\]
    \end{statement}

    \begin{statement}
        \[
            \sum_{k = 0}^{n} (-1)^k C_n^k = 0 \quad (*, x = -1)
        .\]
    \end{statement}

    \begin{statement}
        \[
            \sum_{k = 0}^{n} = \frac{C_n^k}{k + 1}
        .\]
        (Для доказательства надо проинтегрировать бином ньютона).
    \end{statement}

    \section{Числа Фибоначчи}

    \begin{definition}
        \[
            F_1 = F_2 = 1
        .\]
        \[
            F_{n+2} = F_{n+1} + F_n
        .\]
    \end{definition}

    Задача: Количество последовательностей из 0 и 1, не содержащая двух 0 подряд.

    \noindent
    $ P_1 = 2 = F_3 $ \newline
    $ P_2 = 3 = F_4 $ \newline
    $ P_{n+2} = P_n + P_{n+1} = F_{n+4} $ \newline
    Из того, что аналогичную последовательность можно получить отбросив от текущей 1 или 10.

    \begin{statement}
        \[
            F_n = \frac{1}{\sqrt 5}(q^n - q^{-n}), \quad q = \frac{1 + \sqrt 5}{2}.
        .\]
    \end{statement}

    Задача: $ \sum = \left\{ a_1, a_2, a_3 \dots a_n \right\} $ - алфавит.
    Введем ограничения на слова длины $ k $.

    \begin{enumerate}
        \item $ a_1 $ и $ a_2 $ обязательно присутствуют в слове.
        \item $ a_3 $ присутствует на 3-м месте.
        \item $ a_4, a_5 $ стоят рядом в указоном порядке.
        \item Каждая буква встречается в слове не более одного раза.
    \end{enumerate}

    Задача о правильном порядке учета ограничения, наиболее оптимальный - 2, 3, 1, 4. \newline
    Ответ:
    $ 1 \cdot (1 + k - 4) \cdot C_{k-3}^2 \cdot 2 \cdot C_{n-5}^{k-5} \cdot (k - 5)! $

\end{document}

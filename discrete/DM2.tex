\documentclass[a6paper, 10pt]{article}

\usepackage[english, russian]{babel}
\usepackage[T2A]{fontenc}
\usepackage[utf8]{inputenc}
\usepackage{amsmath, amsfonts, amssymb, amsthm, mathtools}
\usepackage{indentfirst}
\usepackage{soulutf8}
\usepackage{geometry}
\usepackage{ulem}
\theoremstyle{plain}

\newtheorem*{theorem}{Th}
\newtheorem{theorem_}{Th}
\newtheorem*{statement}{St}
\newtheorem{statement_}{St}
\newtheorem{corollary}{Cl}[theorem]
\newtheorem{corollary_}{Cl}[theorem_]

\geometry{top=20mm}
\geometry{bottom=20mm}
\geometry{left=10mm}
\geometry{right=10mm}

\newcommand{\N}{\mathbb N}
\newcommand{\Z}{\mathbb Z}
\newcommand{\Q}{\mathbb Q}
\newcommand{\R}{\mathbb R}
\newcommand{\eps}{\varepsilon}
\renewcommand{\phi}{\varphi}
\renewcommand{\kappa}{\varkappa}
\newcommand{\larrow}{\Leftarrow}
\newcommand{\rarrow}{\Rightarrow}
\newcommand{\lrarrow}{\Leftrightarrow}


\begin{document}
	\pagenumbering{gobble}
	\author{Зухба Анастасия Викторовна\\(Конспектировал Иван-Чай)}
	\date{01.09.2023}
	\title{Дискретная математика}

	\linespread{1.4}
	\selectfont

	\maketitle
	\newpage

	\begin{defintion}
	Компонента связности - это связный подграф, максимальный по включению
	\end{defintion}

	\begin{definition}
		\begin{enumerate}
			\item Дерево - связный граф без циклов
			\item Между любыми двумя вершинами $ \exists! $ простой путь
			\item  Связный граф, в котором $ \left| V \right| = \left| E \right| - 1 $
			\item Граф без циклов, в котором $ \left| V \right| = \left| E \right| - 1 $
		\end{enumerate}
	\end{definition}

	\begin{proof}
		(1) -> (2)

		Пусть есть два маршрута между двумя вершинами, тогда есть цикл
	\end{proof}

	\begin{proof}
		(2) -> (3)

		По индукции для количества ребер

		База: $ \left| V \right| = 1 $

		Переход: Пусть $(2) \rarrow (3) $ для $ \left| V \right| \leq k $

		Рассмотрим граф с $ \left| V \right| = k + 1 $

		Удаление ребра приведет к тому, что граф распадется на две компоненты связности
		с $ k_1 $ и $ k_2 $ количеством вершин

		Для каждого из подграфов: $ k_1 + k_2 = k + 1 \rarrow k_1 \leq k \land k_2 \leq k $
		Выполняется (2), а значит, по предположению индукции, выполняется и (3)

		Значит для исходного графа $ 1 + (k_1 - 1) + (k_2 - 1) = k_1 + k_2 - 1 = k $ - ребер
		верно (3)
	\end{proof}


	\begin{proof}
		$ (3) \rarrow (4) $

		Пусть (3) - истинно, но в графе есть цикл(а значит и простой цикл)

		Рассмотрим простой цикл $ U_1, U_2, U_3 \dots U_n, U_1 $ длины $ n $ (тогда имеет и
		$ n $ ребер

		Для каждой вершины $ U_i $ не из цикла рассмотрим кратчайший путь к $ U_1 $
		\begin{statement}
		Первые ребра всех путей различны
		\end{statement}

		Тогда $ \left| E \right| \geq k + n $ - противоречие,
		где $ k $ - количество вершин вне цикла
	\end{proof}

	\begin{proof}
	$ (4) \rarrow (1) $

	Пусть у графа $ k $ компонент связности, тогда каждая компонента
	связности - дерево, а значит для нее верно $ (1) \rarrow (2) \rarrow (3) $

	\begin{allign}
	\sum_{i=1}^k \left| E_i \right| = \sum_{i = 1}^k \left(\left|V_j\right| - 1 \right) =
	\left| V \right| - k \rarrow k = 1 $
	\end{allign}
	\end{proof}

	\begin{definition}
		Эйлеров путь - путь, который проходит по каждому ребру ровно по одному разу
	\end{definition}

	\begin{definition}
		Эйлеров цикл - замакнутый эйлеров путь
	\end{definition}

	\begin{defintion}
		Гамельтонов путь - путь, который проходит через каждую вершину по 1 разу
	\end{defintion}

	\begin{defintion}
		Гамельтонов цикл - замкнутый гамельтонов путь
	\end{defintion}

	\begin{defintion}
		Граф называется эйлеровым, если в нем существет эйлеров цикл
	\end{defintion}

	\begin{theorem}
		В связном графе $ \exists $ эйлеров цикл $ \lrarrow \deg{U} \mod 2 = 0
		\quad \forall U \in V $
	\end{theorem}

	\begin{proof}
	Рассмотрим эйлеров цикл, каждая вершины кроме начала и конца встречается рядом с 2 своими ребрами, ребра не повторяются
	$ \left| E \right| = 1 + 2m + 1 $
	\end{proof}

	\begin{proof}
		Рассмотрим граф, все степени которого четны. В нем всегда будет простой цикл.
		Разобьем его на не пересекающиеся по ребрам простые циклы. Пусть его можно разбить на
		m циклов указанным способом, докажем по индукции для m

		База: $ m = 1 $ - граф содержит эйлеров цикл
		Переход: Предположим, что граф разбивается не более чем
		на $ m $ простых циклов $ \forall m \leq k $, тогда для него $ \exists $
		эйлеров цикл.

		$ m = k + 1 $
		Рассмотрим один из простых циклов $ z = U_1, U_2, U_3 \dots U_i $. Удалим его из графа и
		граф распадется на несколько компонент связности, для каждой из которых
		выполнено предположение индукции, а значит есть эйлеров цикл.
		Для каждой компоненты связности запишем ее эйлеров цикл начиная с вершины,
		встречающейся в цикле $ z $ и "присоеденим" его к обходу z.
	\end{proof}

	\begin{theorem}
		В связном графе $ \exists $ эйлеров цикл $ \lrarrow $ в графе не более двух
		вершин нечетной степени
	\end{theorem}

	\begin{definition}
		Граф $ G(L \lor R, E) называется двудольным, если $ \exists $
		такое разбиение, что $ V = L \lor R, L \land R = \varnothing $, что любое ребро имеет
		вид $ (U, U'): U \in L, U' \in R $
	\end{definition}

	\begin{statement}
		Граф двудольный $ \lrarrow $ в нем нет циклов нечетной длины.
	\end{statement}

	\begin{proof}
		Выберем произвольную вершину и отнесем ее и все вершины на четном расстоянии
		к $ R $, а остальные к $ L $ - это возможно только когда нет циклов нечетной длины
	\end{proof}

	\begin{defintion}
		Паросочетание - это подграф, который является двудольным графом и его степень больше равна
		еденицы
	\end{defintion}

	\begin{theorem}
		Теорема Холла.

		В двудольном графе $ \exists $ паросочетание, включающие все вершины $ L \lrarrow $
		у любого подмножества $ L $ не менее $ \left| V_i \right| $ различных вершин в $ R $

	\end{theorem}
\end{document}

\documentclass[a5paper, 10pt]{article}

\usepackage[english, russian]{babel}
\usepackage[T2A]{fontenc}
\usepackage[utf8]{inputenc}
\usepackage{amsmath, amsfonts, amssymb, amsthm, mathtools}
\usepackage{indentfirst}
\usepackage{soulutf8}
\usepackage{geometry}
\usepackage{ulem}
\usepackage{color}
\usepackage{hyperref}

\theoremstyle{plain}

\newtheorem*{theorem}{Th}
\newtheorem{theorem_}{Th}
\newtheorem*{statement}{St}
\newtheorem{statement_}{St}
\newtheorem{definition}{Def}
\newtheorem*{definition_}{Def}
\newtheorem{exersise}{Ex}
\newtheorem{corollary}{Cl}[theorem]
\newtheorem{corollary_}{Cl}[theorem_]

\geometry{top=20mm}
\geometry{bottom=20mm}
\geometry{left=10mm}
\geometry{right=10mm}

\newcommand{\N}{\mathbb N}
\newcommand{\Z}{\mathbb Z}
\newcommand{\Q}{\mathbb Q}
\newcommand{\R}{\mathbb R}
\newcommand{\eps}{\varepsilon}
\renewcommand{\phi}{\varphi}
\renewcommand{\kappa}{\varkappa}

\newcommand{\larrow}{\leftarrow}
\newcommand{\rarrow}{\rightarrow}
\newcommand{\lrarrow}{\leftrightarrow}
\newcommand{\hrarrow}{\hookrightarrow}
\newcommand{\Larrow}{\Leftarrow}
\newcommand{\Rarrow}{\Rightarrow}
\newcommand{\Lrarrow}{\Leftrightarrow}
\newcommand{\vec}{\overrightarrow}

\hypersetup{
	linktocpage,
    colorlinks=true,
    linktoc=all,
    linkcolor=blue,
}


\begin{document}
	\pagenumbering{gobble}
	\author{\\(Конспектировал Иван-Чай)}
	\date{11.09.2023}
	\title{Доп по линейной алгебре}

	\linespread{1.4}
	\selectfont

	\maketitle
	\newpage

	\tableofcontents

	\section{Векторы}

	\begin{definition}
	Чет про направленные отрезки.
	\end{definition}

	\begin{definition}
	Вектор - класс эквивалентности направленных отрезкв.
	\end{definition}

	Операции
	\begin{itemize}
	\item Сложение.
	\item Умножение на число $ \overrightarrow{b} = \overrightarrow{a} \lambda, \lambda \in \R
		\hrarrow | \overrightarrow{b} | = \lambda |\overrightarrow{a}|. $
	\end{itemize}

	\begin{definition}
	Линейная комбинация векторов $ \vec{a_1}, \vec{a_2} \dots $ - это
	$ \lambda_1 \vec{a_1} + \lambda_2 \vec{a_2} + \dots $.
	\end{definition}

	\begin{definition}
	Линейная оболочка векторов — это множество всех линейных комбинаций данных векторов.
	\end{definition}

	\begin{itemize}
	\item Тривиальная $ \Lrarrow \forall i \lambda_i = 0 $
	\item Нетривиальная $ \Lrarrow \exists i \lambda_i \neq 0 $
	\end{itemize}

	\begin{definition}
	$ a_1, a_2, \dots a_n $ - линейно зависимые, если нетривиальная линейная комбинация, такая что
	\[ % \[
		\sum_{i=1}^n \lambda_i \vec{a_i} = 0(*).
	\] % \]
	\end{definition}
	\begin{definition}
	$ a_1, a_2 \dots a_n $ - линейно независимые, если из (*) $ \Rarrow $ следует тривиальностькомбинации.
	\end{definition}

	Если выписать векторы в матрицу $ A $, и $ \det A = 0 $,
	то они линейно зависимые

	\begin{definition}
	Три вектора комплонарны, если они лежат в одной плоскости.
	\end{definition}

	\begin{statement}
	Три вектора линейно зависимы $ \Lrarrow $ они компланарны.
	\end{statement}

	\begin{definition}
	Упорядоченная совокупность трех(двух) линейно независимых векторов в пространстве(на плоскости)
	называется базисом.
	\end{definition}

	\begin{statement}
		Пусть $ \vec{a_1}, \vec{a_2}, \vec{a_3} $ - базис, тогда $ \forall \vec{v} \hrarrow \exists
		\lambda_1, \lambda_2, \lambda_3 \in \R: \vec{v} =
		\lambda_1 \vec{a_1} +
		\lambda_2 \vec{a_2} +
		\lambda_3 \vec{a_3}. $
	\end{statement}

	\section{Геометрия}

	\begin{definition}
	Система координат - это базис и начало отсчета.
	\end{definition}

\end{document}

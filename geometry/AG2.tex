\documentclass[a5paper, 10pt]{article}

\usepackage[english, russian]{babel}
\usepackage[T2A]{fontenc}
\usepackage[utf8]{inputenc}
\usepackage{amsmath, amsfonts, amssymb, amsthm, mathtools}
\usepackage{indentfirst}
\usepackage{soulutf8}
\usepackage{geometry}
\usepackage{ulem}
\usepackage{color}
\usepackage{hyperref}

\theoremstyle{plain}

\newtheorem*{theorem}{Th}
\newtheorem{theorem_}{Th}
\newtheorem*{statement}{St}
\newtheorem{statement_}{St}
\newtheorem{definition}{Def}
\newtheorem*{definition_}{Def}
\newtheorem{exersise}{Ex}
\newtheorem{corollary}{Cl}[theorem]
\newtheorem{corollary_}{Cl}[theorem_]

\geometry{top=20mm}
\geometry{bottom=20mm}
\geometry{left=10mm}
\geometry{right=10mm}

\newcommand{\N}{\mathbb N}
\newcommand{\Z}{\mathbb Z}
\newcommand{\Q}{\mathbb Q}
\newcommand{\R}{\mathbb R}
\newcommand{\eps}{\varepsilon}
\renewcommand{\phi}{\varphi}
\renewcommand{\kappa}{\varkappa}

\newcommand{\larrow}{\leftarrow}
\newcommand{\rarrow}{\rightarrow}
\newcommand{\lrarrow}{\leftrightarrow}
\newcommand{\hrarrow}{\hookrightarrow}
\newcommand{\Larrow}{\Leftarrow}
\newcommand{\Rarrow}{\Rightarrow}
\newcommand{\Lrarrow}{\Leftrightarrow}

\hypersetup{
	linktocpage,
    colorlinks=true,
    linktoc=all,
    linkcolor=blue,
}


\begin{document}
	\pagenumbering{gobble}
	\author{\\(Конспектировал Иван-Чай)}
	\date{}
	\title{}

	\linespread{1.4}
	\selectfont

	\maketitle
	\newpage

	\tableofcontents

    \section{Ассоциативность умножения матриц}

    \begin{statement}
    A(BC) = (AB)C
    \end{statement}

    \begin{proof}
        Пусть $ A \in \R^{m * n}, B \in \R^{n * p}, c \in \R^{p * q} $. \newline

        Сравним размеры

        $ (AB) \in \R^{m * n}, $
        $ (AB)C \in \R^{m * q} $.

        $ (BC) \in \R^{n * q}, $
        $ A(BC) \in \R^{m * q} $.

        Проверим поэлементно

        $ (AB)_{ij} = \sum_{k=1}^p a_{ik} b_{kj} $.

        $ ((AB)C)_{jl} = $
        $ \sum_{j = 1}^k (AB)_{ij} c_{jl} = $ \newline
        $ = \sum_{j = 1}^k (\sum_{k=1}^p a_{ik} b_{kj}) c_{jl} = $ \newline
        $ = \sum_{j = 1}^k \sum_{k=1}^p a_{ik} b_{kj} c_{jl} = $ \newline
        $ = \sum_{j = 1}^k a_{ik} \sum_{k=1}^p b_{kj} c_{jl} = $ \newline
        $ = (A(BC))_{ij} $.

    \end{proof}

    \section{Свойства векторов}

    \begin{enumerate}
        \item $ \forall \vec{a}, \vec{b} \quad \vec{a} + \vec{b} = \vec{b} + \vec{a} $.
        \item $ \forall \vec{a}, \vec{b}, \vec{c} \quad
            \vec{a} + \left( \vec{b} + \vec{c} \right) =
            \left( \vec{a} + \vec{b} \right) + \vec{c} $.
        \item $ \forall \vec{a} \quad \vec{a} + \vec{0} = \vec{a} $.
        \item $ \forall \vec{a} \quad (-1) \vec{a} $ - противоположный, $ \vec{a} + (- \vec{a}) $.
        \item $ \forall \alpha, \beta \in \R \forall \vec{a} \quad
            (\alpha \beta) \vec{a} = \alpha (\beta \vec{a}) $.
        \item $ \forall \vec{a} \quad 1 \vec{a} = a $.
        \item $ \forall \alpha, \beta \in \R, \forall \vec{a} \quad
            \left( \alpha + \beta \right) \vec{a} = \alpha \vec{a} + \beta \vec{a} $.
        \item $ \forall \alpha \in \R, \forall \vec{a}, \vec{b} \quad
            \alpha \left( \vec{a} + \vec{b} \right) = \alpha \vec{a} + \alpha \vec{b} $.
    \end{enumerate}

    \section{Векторные пространства}

    \begin{definition}
    Множество называется замкнутым относительно операции, если результат этой операции
    над элементами этого множества всегда относится к нему.
    \end{definition}

    \begin{definition}
    Множество векторов, замкнутое относительно линейных операций называется векторным
    пространством.
    \end{definition}

    \begin{definition}
    Если одно векторное пространство является подмножесвтом другого назовем его подпространством.
    \end{definition}

    \begin{definition}
    $ \vec{b} =
        \alpha_1 \vec{a}_1 +
        \alpha_2 \vec{a}_2 +
        \alpha_3 \vec{a}_3
    \dots $ - линейная комбинация векторов.
    \end{definition}

    \begin{definition}
    Линейная комбинация тривиальная, если $ \alpha_i = 0 \quad \forall i $,
    иначе нетривиальная.
    \end{definition}

    \begin{definition}
    Набор векторов $ \vec{a}_1, \vec{a}_2, \dots \vec{a}_n \dots $
    - называется линейно зависимым, если
    $ \exists \alpha_1, \alpha_2, \dots \alpha_n:
        \alpha_1^2 + \alpha_2^2 + \dots \alpha_n^2 > 0 \land
        \sum_{i = 1}^n \alpha_i \vec{a}_i \neq 0 \Lrarrow $
        Существует их нетривиальная комбинация, равная нулю.
    \end{definition}

    \begin{statement}
    Если в наборе есть $ \vec{0} $, то этот набор ЛЗ.
    \end{statement}

    \begin{statement}
    Если набор
    $ \vec{a}_1, \vec{a}_2, \dots \vec{a}_n $ - ЛЗ, то набор \newline
    $ \vec{a}_1, \vec{a}_2, \dots \vec{a}_n \vec{b}_1 \vec{b}_2, \dots \vec{b}_m $
    - ЛЗ.
    \end{statement}

    \begin{statement}
    Если набор ЛНЗ $ \Rarrow $ его любой непустой поднабор тоже линейно независим.
    \end{statement}

    \begin{proof}
        От противного крч.
    \end{proof}

    \begin{theorem}
        Пусть $ \vec{x} =
            \alpha_1 \vec{a}_1 +
            \alpha_2 \vec{a}_2 +
            \alpha_3 \vec{a}_3
        \dots $ - единственное разложение $ \vec{x} $ $ \Rarrow $ набор $
            \vec{a}_1,
            \vec{a}_2,
            \vec{a}_3
        \dots $ - ЛНЗ.
    \end{theorem}

    \begin{proof}
        $
        (\alpha_1 - \beta_1) \vec{a}_1 +
        (\alpha_2 - \beta_2) \vec{a}_2 +
        (\alpha_3 - \beta_3) \vec{a}_3 \dost = \vec{0} $

        Обратное: Допустим $ \exists
            \gamma_1,
            \gamma_2 \dots: \\
            \gamma_1 \vec{a}_1 +
            \gamma_2 \vec{a}_2 \dots = \vec{0}, $ тогда $
            (\gamma_1 + \alpha_1) \vec{a}_1 +
            (\gamma_2 + \alpha_2) \vec{a}_1 = \vec{x} $ - противоречие.
    \end{proof}

    \begin{theorem}
    Набор ЛЗ $ \Lrarrow $
    $ \exists $ вектор из набора равный линейной комбинации остальных.
    \end{theorem}

    \begin{proof}
    $ \exists \alpha_1, \alpha_2 \dots > 0:
    \alpha_1 \vec{a}_1 + \dots = \vec{0} $, тогда
    $ \vec{a}_1 =
    - \frac{\alpha_2}{\alpha_1} \vec{a_2}
    - \frac{\alpha_3}{\alpha_1} \vec{a_3}
    \dots $.

    Там еще как-то в обратную было.
    \end{proof}

    \begin{statement}
    Система из одного вектора - ЛЗ $ \Lrarrow $ он нулевой.
    \end{statement}

    \begin{statement}
    Система из 2 векторов ЛЗ $ \Lrarrow $ они колониарны.
    \end{statement}

    \begin{statement}
    Система из 3 векторов ЛЗ $ \Lrarrow $ они компланарны.
    \end{statement}

    \begin{statement}
    Система из 4 векторов ЛЗ всегда.
    \end{statement}

    \begin{proof}
    $ \vec{a}, \vec{b}, \vec{c} $ - компланарны
    \begin{enumerate}
        \item $ \vec{a} + \vec{b} $ - колониарны $ \Lrarrow $ они лз.
        \item $ \vec{a} + \vec{b} $ - неколониарны $ \Rarrow \quad
            \exists \alpha, \beta: \alpha \vec{a} + \beta \vec{b} = \vec{c} $ - ЛЗ.
    \end{enumerate}

    В обратку наоборот.
    \end{proof}
\end{document}

\documentclass[a5paper, 10pt]{article}

\usepackage[english, russian]{babel}
\usepackage[T2A]{fontenc}
\usepackage[utf8]{inputenc}
\usepackage{amsmath, amsfonts, amssymb, amsthm, mathtools}
\usepackage{indentfirst}
\usepackage{soulutf8}
\usepackage{geometry}
\usepackage{ulem}
\usepackage{color}
\usepackage{hyperref}

\theoremstyle{plain}

\newtheorem*{theorem}{Th}
\newtheorem{theorem_}{Th}
\newtheorem*{statement}{St}
\newtheorem{statement_}{St}
\newtheorem{definition}{Def}
\newtheorem*{definition_}{Def}
\newtheorem{exersise}{Ex}
\newtheorem{corollary}{Cl}[theorem]
\newtheorem{corollary_}{Cl}[theorem_]

\geometry{top=20mm}
\geometry{bottom=20mm}
\geometry{left=10mm}
\geometry{right=10mm}

\newcommand{\N}{\mathbb N}
\newcommand{\Z}{\mathbb Z}
\newcommand{\Q}{\mathbb Q}
\newcommand{\R}{\mathbb R}
\newcommand{\eps}{\varepsilon}
\renewcommand{\phi}{\varphi}
\renewcommand{\kappa}{\varkappa}

\newcommand{\larrow}{\leftarrow}
\newcommand{\rarrow}{\rightarrow}
\newcommand{\lrarrow}{\leftrightarrow}
\newcommand{\hrarrow}{\hookrightarrow}
\newcommand{\Larrow}{\Leftarrow}
\newcommand{\Rarrow}{\Rightarrow}
\newcommand{\Lrarrow}{\Leftrightarrow}

\hypersetup{
	linktocpage,
    colorlinks=true,
    linktoc=all,
    linkcolor=blue,
}


\begin{document}
	\pagenumbering{gobble}
	\author{Тюленев Александр Иванович\\(Конспектировал Иван-Чай)}
	\date{08.09.2023}
	\title{Матанализ}

	\linespread{1.4}
	\selectfont

	\maketitle
	\newpage

	\tableofcontents

	\section{Какое-то доказательство}

	Здесь должно быть доказательство того, что из леммы Кантора и леммы архимеда следует
	аксиома непрерывности, но оно комплексное и ненулевая действительная часть появится,
	тогда и только тогда, когда выйдет запись лекции на лектории фпми. Прошу понять и простить.

	\section{Счетные и несчетные множества}

	\begin{definition}
		Отображение $ f: X \rightarrow Y $ называется биекцией из X на Y, если оно
		и сюрьекция и инъекция $ \Lrarrow $ оно обратимо.
	\end{definition}

	\begin{definition}
		Множество $ X $ называется конечным, если $ \exists n \in \N $ и биекция
		$ X $ на $ \left\{ 1, \dots n \right\} $, в противном случае оно называется
		бесконечным.
	\end{definition}

	\begin{definition}
		Будем говорить, что $ X $ и $ Y $ равномощны, если $ \exists $ биекция $ X $ на $ Y $.

		Обозначим равномощность множеств $ A $ и $ B $, как $ A \lrarrow B $.
	\end{definition}


	\begin{definition}
		Будем говорить, что мощность множествва $ Y $ не меньше мощности $ X $, если
		$ \exists Y' \subset Y: Y $ и $ X $ равномощны.
	\end{definition}

	\begin{definition}
		Множество $ X $ называется счетным, если оно бесконечно и $ X $ равномощно $ \N $.
	\end{definition}

	\begin{definition}
		Множество $ X $ называется несчетным, если оно бесконечно и не равномощно $ \N $.
	\end{definition}

	\begin{theorem}
		$ \Q $ - счетно.
	\end{theorem}

	\begin{proof}
		"Разместим" все рациональные числа в "таблице"

		\begin{tabular}{|l| |*{6}{c}}
			\N \setminus \Z
					   & 0 & 1               & -1               & 2 & -2 & \dots \\ \hline
			1          & 0 & 1               & -1               & 2 & -2 & \dots \\
			2          & 0 & $ \frac{1}{2} $ & $ -\frac{1}{2} $ & 1 & -1 & \dots \\
			3          & 0 & $ \frac{1}{3} $ & $ -\frac{1}{3} $ &
				$ \frac{2}{3} $ & $ -\frac{2}{3} $ & \dots \\
				\vdots     &  \vdots & \vdots  & \vdots & \vdots & \vdots & \ddots
		\end{tabular}

		Будем двигаться "по змейке", при этом пропуская повторяющиеся числа.
		За счет этого получим инъекцию из $ \N $ в $ Q $.
		Т.к. $ \forall a = \frac{p}{q} $ найдется квадрат в котором есть это
		число змейка в него попадет $ \Rarrow $ это сюрьекция.
	\end{proof}


	\begin{theorem}
		Множество $ \R $ - несчетно
	\end{theorem}

	\begin{proof}
		$ \R $ бесконечно, поскольку содержит $ \N $, покажем, что $ \neg \N \lrarrow \R $

		Допустим, что \exists \N \lrarrow ^x \R, x(n) \equiv x_n $

		Построим последовательность $ J_k \subset J_{k-1} \dots J_3 \subset J_2 \subset J_1:
		\forall k \in \N \quad x_k \notin J_k \Rarrow \{x_1, x_2 \dots x_k \} \cap J_k = \varnothing
		\forall k \in \N $
	\end{proof}

	\begin{exersise}
		Доказать $ \N \lrarrow \Z, \N \lrarrow \Q $.
	\end{exersise}

	\begin{exersise}
		Докозать, что
		$ \left[0, 1\right] \lrarrow
		\left(0, 1\right] \lrarrow
		\left[0, 1\right) \lrarrow
		\left(0, 1\right) \lrarrow \R $.
	\end{exersise}

	\section{Последовательности}

	\begin{definition}
		Последовательностью будем называть отображение $ x : $ \N \rightarrow \R $
		При этом $ x\left(n\right) \equiv X \quad \forall n \in \N $.
	\end{definition}

	\begin{definition}
		Элементом последовательности называется пара $ \left(n, x_n\right) $.
	\end{definition}

	\begin{definition}
		При этом числа $x_n, n \in \N $ называют значениями последовательности.
	\end{definition}

	\begin{definition}
		Вся последовательность обозначается $ \{x_n\} \equiv \{x_n\}_{n=1}^\infty.
	\end{definition}

	\begin{definition}
		$ \widehat{\R} := \overline{\R} \cap \{ \infty \} $.
	\end{definition}

	\begin{definition}
		Если $ a \in \R $, то $U(a) = (a - \eps, a + \eps) $.
	\end{definition}

	\begin{definition}
		$ U(+\infty) := ( \frac{1}{\eps}, +\infty ) $.
	\end{definition}

	\begin{definition}
		$ U(-\infty) := ( -\frac{1}{\eps}, -\infty ) $.
	\end{definition}

	\begin{definition}
		$ U(\infty) := ( -\frac{1}{\eps}, -\infty ) \cap ( \frac{1}{\eps}, +\infty ) $.
	\end{definition}

	\section{Пределы}

	\begin{definition}
		Будем говорить, что элемент $ a \in \widehat{\R} $ является пределом числовой
		последовательности $ \{ x_n \} $ и писать $ \lim_{n \to \infty}X_n = a$ , если
		$ \forall \eps > 0 \exists N = N(\eps) \in \N: \forall n \geq N(\eps) \hrarrow x_n \in
		U_{\eps}(a) $.
	\end{definition}


	\begin{statement}
		Пример:
		\[ lim_{n \to \infty} \frac{1}{n} = 0. \]
	\end{statement}

	\begin{proof}
		\begin{allign}
			\forall \eps > 0 \exists N(\eps) = \left[ \frac{1}{\eps} \right] + 1 \in \N: \\
			\forall \quad n \geq
			N(\eps) \hrarrow \frac{1}{n}  \leq \frac{1}{N(\eps)} \leq
				\frac{1}{\left[ \frac{1}{\eps} \right] + 1} \leq \frac{1}{ \frac{1}{\eps} } = \eps
		\end{allign}.
	\end{proof}
\end{document}

\documentclass[a5paper, 10pt]{article}

\usepackage[english, russian]{babel}
\usepackage[T2A]{fontenc}
\usepackage[utf8]{inputenc}
\usepackage{amsmath, amsfonts, amssymb, amsthm, mathtools}
\usepackage{indentfirst}
\usepackage{soulutf8}
\usepackage{geometry}
\usepackage{ulem}
\usepackage{color}
\usepackage{hyperref}

\theoremstyle{plain}

\newtheorem*{theorem}{Th}
\newtheorem{theorem_}{Th}
\newtheorem*{statement}{St}
\newtheorem{statement_}{St}
\newtheorem{definition}{Def}
\newtheorem*{definition_}{Def}
\newtheorem{exersise}{Ex}
\newtheorem{corollary}{Cl}[theorem]
\newtheorem{corollary_}{Cl}[theorem_]

\geometry{top=20mm}
\geometry{bottom=20mm}
\geometry{left=10mm}
\geometry{right=10mm}

\newcommand{\N}{\mathbb N}
\newcommand{\Z}{\mathbb Z}
\newcommand{\Q}{\mathbb Q}
\newcommand{\R}{\mathbb R}
\newcommand{\eps}{\varepsilon}
\renewcommand{\phi}{\varphi}
\renewcommand{\kappa}{\varkappa}

\newcommand{\larrow}{\leftarrow}
\newcommand{\rarrow}{\rightarrow}
\newcommand{\lrarrow}{\leftrightarrow}
\newcommand{\hrarrow}{\hookrightarrow}
\newcommand{\Larrow}{\Leftarrow}
\newcommand{\Rarrow}{\Rightarrow}
\newcommand{\Lrarrow}{\Leftrightarrow}

\hypersetup{
    linktocpage,
    colorlinks=true,
    linktoc=all,
    linkcolor=blue,
}


\begin{document}
    \pagenumbering{gobble}
    \author{
        Тюленев А.И.\\
        Конспектировал Иван-Чай\\
        Если будут ошибки дайте пизды @coolstory\_bob}
    \date{06.09.2023}
    \title{Введение в математический анализ}

    \linespread{1.4}
    \selectfont

    \maketitle
    \newpage

    \tableofcontents

    \begin{definition}
        Расширенная числовая прямая
        $ \overline{\R} := \R \cap \{+\infty\} \cap \{-\infty\} $.
    \end{definition}

    \section{Верхняя и нижняя грань множества}

    \begin{definition}
    Число $ M $ называется верхней гранью числового (непустого) множества $ A \subset \R $,
    если $ a \leq M \quad \forall a \in A $.
    \end{definition}

    \begin{definition}
    Число $ M $ называется нижней гранью числового (непустого) множества $ A \subset \R $,
    если $ a \geq M \quad \forall a \in A $.
    \end{definition}

    \section{Супремум}

    \begin{definition}
    Пусть $ A \subset \R $ - ограниченное сверху множество. Число $ M \in \R $ называется
    супремумом $ A $ и записывается $ M = \sup A $, если

    \begin{itemize}
        \item $ M $ - является верхней гранью $ A \Lrarrow \forall a \in A \hrarrow a \leq M $.
        \item $ \forall M' < M \quad \exists a(M') \in A : M' < a(M') \leq M $.
    \end{itemize}
    \end{definition}

    \begin{definition}
    Если $ A \subset \R $ - неограниченно сверху, то $ \sup A := +\infty $.
    \end{definition}

    \section{Теорема о существовании и единственности супремума}

    \begin{theorem}[Теорема о существовании и единственности супремума]

    \[ \forall A \subset \R, A \neq \varnothing \quad \exists! \sup A \]
    \end{theorem}

    \begin{proof}
        Для неограниченного сверху $ A \subset \R \hrarrow \sup A := +\infty $

        Для ограниченного сверху $ A \subset \R \hrarrow \quad \exists $ верхняя грань

        Пусть $ B = \{ M \in \R: M - $ верхняя грань $ \} $, тогда
        $ B \neq \varnothing \quad \land $
        $ A $ расположенно левее $ B $

        По аксиоме непрерывности $ \exists c \in \R: a \leq c \leq M \quad
        \forall a \in A, \forall M \in B $

        Покажем, что $ c = \sup A: $

        \begin{enumerate}
            \item Т.к. $ a \leq c \quad \forall a \in A \Rarrow c $ - верхняя грань.
            \item Предположим, что $ \exists c' < c: c' $ - верхняя грань $ A $.
                Тогда $ c' \in B $, но $ c $ было выбрано так, что $ c \leq M \quad \forall M \in B $,
                в частности $ c \leq c' $ - противоречие.
                $ \Rarrow \forall c' < c \hrarrow
                    c' \notin B \Lrarrow
                    \neg \left(c' \in B \right) \Lrarrow
                    \neg(\forall a \in A \hrarrow a \leq c') \Lrarrow
                    \exists a(c') \in A: a(c') > c' $,
                    но т.к. $ a(c') \in A $, то $ a(c') \leq c $.
                    Т.е. $ \forall c' < c \quad \exists a(c') \in A: c' < a(c') \leq c $
        \end{enumerate}

        Единственность:

        Допустим $ \exists M_1 \in \R: M_1 = \sup A \quad \land \exists M_2 \in \R: M_2 = \sup A $.
        Пусть $ M_1 > M_2 $, но тогда по 2 пункту определения супремума для $ M_1 $:
        $ \exists a(M_2) \in A: a(M_2) > M_2 \Rarrow $ $ M_2 $ - не верхняя грань, противоречие.
    \end{proof}

    \begin{statement}
        $ M = \sup A \Lrarrow $
        \begin{cases}
            M \in \overline \R $ \\
            a \leq M \quad \forall a \in A \\
            \forall M' < M \exists a(M') \in A: M' < a(M') \leq M \\
        \end{cases}
    \end{statement}

    \section{Инфинум}

    \begin{definition}
    $ m \in \R $ - называется инфинумом ограниченного снизу множества, если

    \begin{cases}
        m \leq a \quad \forall a \in A \\
        \forall m' > m \quad \exists a(m') \in A: m \leq a(m') < m'
    \end{cases}
    \end{definition}

    \begin{definition}
        Если $ A $ - неограниченно снизу, то $ \inf A := - \infty $.
    \end{definition}

    \begin{theorem}[Теорема о существовании и единственности инфинума]
        \[ \forall A \subset \R, A \neq \varnothing \quad \exists! \inf A \]
    \end{theorem}

    \begin{statement}
    $ m = \inf A $
    \begin{cases}
        m \in \overline \R \\
        m \leq a \quad \forall a \in A \\
        \forall m' > m \quad \exists a(m') \in A: m \leq a(m') < m' \\
    \end{cases}
    \end{statement}

    \begin{statement}
        Аксиома непрерывности $ \Lrarrow $
            Теорема о существовании и единственности супремума и
            Теорема о существовании и единственности инфинума.
    \end{statement}

    \section{Лемма архимеда}

    \begin{theorem}[Лемма Архимеда]
    \[ \forall M' \in \R \quad \exists N(M') \in \N: N(M') > M' \]
    \end{theorem}

    \begin{proof}
    Предположим, что $ \N $ - ограниченно сверху $ \Rarrow $ $ \exists $ верхняя грань и конечный
    супремум $ M = \sup \N < + \infty $
    $ \Rarrow M' = M - 1: \exists N(M') \in \N: N(M') > M - 1 \Rarrow
    N(M') + 1 > M $ - противоречие.
    \end{proof}

    \section{Лемма Кантора}

    \begin{definition}
    Отображение из $ \N $ в множество всех отрезков на числовой прямой $ \R $ назовем
    последовательностью отрезков и обозначим
    $ \left\{ \left[ a_n, b_n \right] \right\}_{n=1}^{+\infty} $.
    \end{definition}

    \begin{definition}
    Будем говорить, что
    $ \left\{ \left[ a_n, b_n \right] \right\}_{n=1}^{+\infty} $
    - последовательность вложенных отрезков, если
    $
    \left[ a_{n+1}, b_{n+1} \right]
    \subset
    \left[ a_{n}, b_{n} \right]
    \forall n \in \N
    $.
    \end{definition}

    \begin{theorem}[Лемма Кантора или принцип вложенных отрезков]
        $ \forall $ последовательности вложеннных отрезков
    $
    \left\{ \left[ a_n, b_n \right] \right\}_{n=1}^{+\infty} \hrarrow
    \exists x \in
    \bigcap\limits_{n=1}^{+\infty}
    \left[ a_{n}, b_{n} \right]
    \Lrarrow
    \bigcap\limits_{n=1}^{+\infty}
    \left[ a_{n}, b_{n} \right]
    \neq \varnothing
    $.
    \end{theorem}

    \begin{proof}
    Справедливо неравенство
    $ -\infty < a_n \leq a_{n+1} \dots \leq b_{n+1} \leq b_n < +\infty \quad \forall n \in \N $.

    $ \forall m, n \in \N \hrarrow -\infty < a_m \leq b_n < +\infty $.

    $ m \geq n $
    $ \Rarrow $ по индукции $ b_m \leq b_n \Rarrow a_m \leq b_m \leq b_n $.

    $ n < m $
    $ \Rarrow a_m \leq a_n \leq b_n $.

    \noindent
    $ A := \{a_1, a_2, a_3 \dots \} $.
    \newline
    $ B := \{b_1, b_2, b_3 \dots \} $.

    Из того, что $ a_m \leq b_n \quad \forall m, n \in \N $
    вытекает, что $ A $ расположенно левее $ B $
    $ \Rarrow \exists c \in \R: a_n \leq c \leq b_n \quad \forall n, m \in \N $
    $ \Rarrow a_n \leq c \leq b_n \quad \forall n \in \N $
    $ \Rarrow c \in \left[ a_n, b_n \right] \quad \forall \N $
    $ \Rarrow c \in $
    $ \left\{ \left[ a_n, b_n \right] \right\}_{n=1}^{+\infty} $.
    \end{proof}

    \begin{exersise}
    Доказать лемму Кантора без аксиомы непрерывности с использованием супремума и инфинума.
    \end{exersise}

    \section{Стягивающяяся система вложенных отрезков}

    \begin{definition}
    Последовательность вложенных отрезков
    $ \left\{ \left[ a_n, b_n \right] \right\}_{n=1}^{+\infty} $
    называется стягивающейся, если
    $ \forall n \in \N \quad \exists $
    отрезок
    $ \left[ a_{m(n)}, b_{m(n)} \right]: $
    $ \left| \left[ a_{m(n)}, b_{m(n)} \right] \right| < \frac{1}{n} $.
    \end{definition}

    \begin{theorem}
    Стягивающяяся последовательность вложенных отрезков имеет единственную общую точку.
    $ \Lrarrow $
    $ \exists! x \in $
    $ \bigcap\limits_{n=1}^{+\infty} $
    $ \left\{ \left[ a_n, b_n \right] \right\}_{n=1}^{+\infty} $.
    \end{theorem}

    \begin{proof}
    Предположим, что $ \exists x_1, x_2: $
    \begin{cases}
        x_1 \in \bigcap\limits_{n=1}^{+\infty}
        \left\{ \left[ a_n, b_n \right] \right\}_{n=1}^{+\infty} \\
        x_2 \in \bigcap\limits_{n=1}^{+\infty}
        \left\{ \left[ a_n, b_n \right] \right\}_{n=1}^{+\infty} \\
        x_1 \neq x_2 \\
    \end{cases}

    \noindent
    $ x_1 \neq x_2 \Rarrow |x_1 - x_2| > 0 $.
    \newline
    $ \frac{1}{M} := |x_1 - x_2| $.

    По лемме архимеда
    $ \exists N \in \N: N > M \Rarrow \frac{1}{N} < |x_1 - x_2| $
    $ \Rarrow $ в силу того, что система отрезков стягивающяяся
    $ \exists $
    $ \left[ a_{m(n)}, b_{m(n)} \right]: $
    $ \left| \left[ a_{m(n)}, b_{m(n)} \right] \right| < \frac{1}{n} $.
    В частности $ x_1, x_2 \in $
    $ \left[ a_{m(n)}, b_{m(n)} \right] \Rarrow \left| x_1 - x_2 \right| < \frac{1}{n}$
    - противоречие.
    \end{proof}

    \begin{theorem}[3 принципа непрерывности числовой прямой]
    Следующие утверждения эквивалентны:

    \begin{itemize}
        \item Аксиома непрерывности.
        \item $ \exists \inf A, \exists \sup A \quad \forall A \neq \varnothing $.
        \item Лемма Кантора и лемма Архимеда.
    \end{itemize}
    \end{theorem}

    \begin{statement}
    Лемма Кантора может не работать для интервалов.
    \end{statement}

    \begin{statement}
    Пример: \newline
        $ a_n = 0 \quad \forall n \in \N $ \newline
        $ b_n = \frac{1}{n} \quad \forall n \in \N $ \newline
        $ (a_n, b_n) = \left( 0, \frac{1}{n} \right) $ \newline
        $ \bigcap\limits_{n=1}^{+\infty} $
        $ \left( 0, \frac{1}{n} \right) $
        $ = \varnothing $
    \end{statement}

    \begin{proof}
    Предположим $ \exists x > 0: $ \newline
        $ x \in \bigcap\limits_{n=1}^{+\infty} $
        $ \left( 0, \frac{1}{n} \right) $
        $ \Rarrow n < \frac{1}{x} \quad \forall n \in \N $
        - противоречие с лемой архимеда.
    \end{proof}
\end{document}

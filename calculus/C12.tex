% /*
%  * ----------------------------------------------------------------------------
%  * "THE BEER-WARE LICENSE" (Revision 42):
%  * @iv4n-t3a wrote this file.  As long as you retain this notice you
%  * can do whatever you want with this stuff. If we meet some day, and you think
%  * this stuff is worth it, you can buy me a beer in return.   Иван-Чай
%  * ----------------------------------------------------------------------------
%  */

\documentclass[a5paper, 10pt]{article}

\usepackage[english, russian]{babel}
\usepackage[T2A]{fontenc}
\usepackage[utf8]{inputenc}
\usepackage{amsmath, amsfonts, amssymb, amsthm, mathtools}
\usepackage{indentfirst}
\usepackage{soulutf8}
\usepackage{geometry}
\usepackage{ulem}
\usepackage{color}
\usepackage{hyperref}

\theoremstyle{plain}

\newtheorem{theorem}{Th}
\newtheorem*{theorem_}{Th}
\newtheorem*{statement}{St}
\newtheorem{statement_}{St}
\newtheorem{definition}{Def}
\newtheorem*{definition_}{Def}
\newtheorem{lemma}{Lem}
\newtheorem*{lemma_}{Lem}
\newtheorem*{note}{Nt}
\newtheorem{exersise}{Ex}
\newtheorem{corollary}{Cl}[theorem]
\newtheorem{corollary_}{Cl}[theorem_]

\geometry{top=20mm}
\geometry{bottom=20mm}
\geometry{left=10mm}
\geometry{right=10mm}

\newcommand{\N}{\mathbb N}
\newcommand{\Z}{\mathbb Z}
\newcommand{\Q}{\mathbb Q}
\newcommand{\R}{\mathbb R}
\newcommand{\eps}{\varepsilon}
\renewcommand{\phi}{\varphi}
\renewcommand{\kappa}{\varkappa}
\renewcommand{\vec}{\overrightarrow}

\newcommand{\oR}{\overline{\mathbb R}}
\newcommand{\hR}{\hat{\mathbb R}}
\newcommand{\Ud}{\dot{U}}

\newcommand{\larrow}{\leftarrow}
\newcommand{\rarrow}{\rightarrow}
\newcommand{\lrarrow}{\leftrightarrow}
\newcommand{\hrarrow}{\hookrightarrow}
\newcommand{\Larrow}{\Leftarrow}
\newcommand{\Rarrow}{\Rightarrow}
\newcommand{\Lrarrow}{\Leftrightarrow}

\hypersetup{
	linktocpage,
    colorlinks=true,
    linktoc=all,
    linkcolor=blue,
}


\begin{document}
	\pagenumbering{gobble}
	\author{Тюленев Александр Иванович\\(Конспектировал Иван-Чай)}
	\date{12 лекция}
	\title{Введение в математический анализ}

	\linespread{1.4}
	\selectfont

	\maketitle
	\newpage

	\tableofcontents

    \section{Предел композиции}


    \begin{theorem_}[Предел композиции 2]
        Пусть $ f: U_{\sigma_0} \to \R, y : \Ud_{\delta_0}(x_0) \to U_{\sigma_0}(y_0) $.
        Пусть $ f $ непрерывна в $ y_0 $ и $ \lim\limits_{x \to x_0}  y(x) = y_0 $
        Тогда $ \exists \lim\limits_{x \to x_0} f_0 y(x) = \lim\limits_{x \to x_0} f(y(x)) = f(y_0) $
    \end{theorem_}

    \begin{proof}
        . \newline
        \begin{cases}
            \forall \eps > 0 \quad \exists \sigma(\eps) \in (0, \delta_0) \quad \forall
            y \in U_{\delta(\eps)}(y_0) \hrarrow |f(y) - f(y_0)| < \eps \\
            \forall \sigma > 0 \quad \exists \delta(\eps) \in (0, \delta_0) \quad \forall
            x \in U_{\delta(\eps)}(x_0) \hrarrow |y(x) - y_0| < \eps \\
        \end{cases}
        \[
            \forall \eps > 0 \quad \exists  \widetilde{\delta} = \delta(\sigma(\eps)) \in (0, \delta_0)
        .\] \[
            \forall x \in \Ud
        .\]

        Там дальше было надо дотехать будет
    \end{proof}

    \section{Теорема об обратной функции}

    \begin{lemma}
        $ f: X \to Y $ - обратима на $ X \Lrarrow f $ - сюрьекция $ \land f $ - инъекция.
    \end{lemma}

    \begin{proof}[$ \Larrow $]
        Пусть $ f $ - инъекция и сюръекция. \newline
        Рассмотрим $ y \in Y $, т.к. $ f $ - сюрьекция. \newline
        Но т.к. f - инъекция $ x $ - единственный $ \Rarrow f^{-1}(y) = x $(единств)
        $ \exists x \in X: f(x) = y $
    \end{proof}

    \begin{proof}[$ \Rarrow $]
        Т.к. f обратимо, то $ \exists f^{-1}: Y \to X \Rarrow f - $ сюрьекция.
        \[
            \forall y \in Y \quad \exists x =  f ^{-1}(y): f(f ^{-1}(y)) = f(x) = y
        .\]

        Покажем, что $ f $ - инъекция.
        Неуспел затехать, сори
    \end{proof}

    \begin{lemma}
        Пусть $ X \subset \R, X \neq \varnothing $. \newline
        Пусть $ f: X \to \R $ - строго монотонна \newline
        Тогда $ f $ - обратима и $ f^{-1} $ строго возрастает, если $ f $ возрастает,
        иначе убывает, если $ f $ возрастает,
        иначе убывает.
    \end{lemma}

    \begin{proof}
        В силу предыдущей доказаной леммы достаточно доказать, что $ f: X \to f(x) $ - инъективно,
        но это следует из строгой монотонности.

        Рассморим случай строгого возрастания на $ X $, т.к. случай убывания аналогичен.
        $ f^{-1}: f(x) \to X $ сущ в силу инъективности $ f $.
        Покажем, что она строго возрастает.

        $ y_1, y_2 \in f(x) $
        Пусть $ y_2 > y_1 $
        покажем, что $ f ^{-1}(y_2) > f ^{-1}(y_1) $.
        Предположим противное.
        Т.к. $ f $ строго возрастает $ y_2 = f(f ^{-1}(y_2)) \leq f(f ^{-1}(y_1)) = y_1 \Rarrow
        y_2 < y_1 $
    \end{proof}

    \begin{theorem_}[Теорема об обратной функции]
        Пусть $ f \in C([a, b]) $ - строго монотонна на $ [a, b] $.
        Тогда $ \exists f ^{-1} \in C([m, M]) $ имеет характер монотонности тот же, что и $ f $.
        $ m = \min\limits_{x \in [a, b] } f(x) $,
        $ M = \max\limits_{x \in [a, b] } f(x) $
    \end{theorem_}

    \begin{proof}
        Тот факт, что $ \exists f^{-1} $ имеет тот же характер монотонности, что и $ f $
        вытекает из предыдущей леммы.
        Осталось показать непрерывность $ f ^{-1} $ в любой точке $ y \in [m, M] $.

        Рассмотрим случай $ y_0 \in (m, M) $

        Т.к. $ y_0 \in (m, M) $, то $ x_0 \in (a, b) $ \newline
        Фиксируем $ \eps > 0: U_{\eps}(x_0) \in (a, b) $

        Рассмотрим отрезок $ [x_0 - \eps, x_0 + \eps] \subset (a, b) $.
        $ f $ - строго возрастает и непрерывно $ \Rarrow f $ -
        осуществляет биекцию $ [x_0 - \eps, x_0 + \eps] $ на $ [f(x_0-\eps),f(x_0+\eps)]$.
        $ \delta(\eps) = min \{
            f(x_0) - f(x_0-\eps),
            f(x_0+\eps) - f(x_0)
        \} $.
        Рассмотрим интервал $ (f(x_0) - \delta(\eps), f(x_0) + \delta(\eps)) \subset
        (f(x_0-\eps), f(x_0+\eps))$
        \[(f(x_0) - \delta(\eps), f(x_0) + \delta(\eps)) \subset
        (f(x_0-\eps), f(x_0+\eps))
        .\] \[ \Downarrow \] \[
            \forall y \in (f(x_0) - \delta(\eps), f(x_0) + \delta(\eps)) \hrarrow
            f^{-1}(y) \in U_{\eps}(x_0)) = U_{\eps}(f ^{-1}(y_0))
        .\]
    \end{proof}

    \begin{corollary_}
        Пусть $ f \in C((a, b)) $ и строго монотонна.
        Тогда $ \exists f ^{-1} \in C((m, M)) $ и строго монотонна с тем же характером, что и f $

        \[
            m = \inf\limits_{x \in (a, b)} f(x) \in \oR
        .\] \[
            M = \sup\limits_{x \in (a, b)} f(x) \in \oR
        .\]

        Т.к. $ f $ - строго монотонна, то $ \exists f ^{-1} $ имеющая тот же характер монотонности.
        Покажем, что $ f((a, b)) = (m, M) $

        В силу обобщения о промежуточном значении $ (m, M) \subset f((a, b)) $, но m и M
        не принимаются.

        Действительно, если $ \exists x^* \in (a, b): M = f(x^*) \Rarrow
        \exists x^{**} \in (x^*, b): f(x^{**}) > f(x^*) = M $ - противоречие.

        Отсюда $ f((a, b)) \subset (m, M) \Rarrow f((a, b)) = (m, M)$.
        Далее непрерывность доказывается как в предыдущей теореме.
    \end{corollary_}

    \section{Непрерывность элементарных функций и первый замечательный предел}

    \begin{definition}
        Длина кривой - супремум множества длин всех вписаных в нее ломаных, но мы этого пока
        не знаем.
    \end{definition}

    \begin{lemma}
        $ \sin{x} < x < \tg{x} \quad \forall \pi \in \left(0, \frac{\pi}{2}\right) $.
    \end{lemma}

    \begin{proof}
        Ну там кружочки, треугольнички порисуйте и методом площадей, я не в силах это затехать.
    \end{proof}

    \begin{theorem_}[Первый замечательный предел]
        \[
            \exists \lim\limits_{x \to 0} \frac{\sin{x}}{x}= 0
        .\]
    \end{theorem_}

    \begin{proof}
        В силу принципа локализации рассмотрим на
        $ \left( - \frac{\pi}{2}, \frac{\pi}{2} \right) \setminus \{ 0 \} $

        Для x > 0:
        \[
            \frac{\sin{x}}{\tg{x}} < \frac{\sin{x}}{x} < 1
            .\] \[ \Downarrow \] \[
            \cos{x} < \frac{\sin{x}}{x} < 1
        .\]

        Для в силу четности:
            $ \cos{x} < \frac{\sin{x}}{x} < 1 $ на
        $ \left( - \frac{\pi}{2}, \frac{\pi}{2} \right) \setminus \{ 0 \} $
        $ \Rarrow \exists \lim\limits_{x \to x_0} \frac{\sin{x}}{x}= 1 $
    \end{proof}

    \begin{theorem}
        $ \sin{x}, \cos{x} $ - непрерывны.
    \end{theorem}

    \begin{proof}
        Докажем для $ \sin{x} $, т.к. $ \cos{x} = \sin{\left( x + \frac{\pi}{2} \right) } $

        \[
        \left| \sin{x_1} - \sin{x_2} \right| =
    \left| 2 \sin{ \frac{x_1-x_2}{2}} \cos{ \frac{x_1 + x_2}{2}} \right|
    \leq 2 \left| \sin{ \frac{x_1 - x_2}{2}} \right| \leq
    2 \frac{|x_1-x_2|}{2} = \left| x_1-x_2 \right|
        .\]
    \end{proof}
\end{document}

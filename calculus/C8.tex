% /*
%  * ----------------------------------------------------------------------------
%  * "THE BEER-WARE LICENSE" (Revision 42):
%  * @iv4n-t3a wrote this file.  As long as you retain this notice you
%  * can do whatever you want with this stuff. If we meet some day, and you think
%  * this stuff is worth it, you can buy me a beer in return.   Иван-Чай
%  * ----------------------------------------------------------------------------
%  */

\documentclass[a5paper, 10pt]{article}

\usepackage[english, russian]{babel}
\usepackage[T2A]{fontenc}
\usepackage[utf8]{inputenc}
\usepackage{amsmath, amsfonts, amssymb, amsthm, mathtools}
\usepackage{indentfirst}
\usepackage{soulutf8}
\usepackage{geometry}
\usepackage{ulem}
\usepackage{color}
\usepackage{hyperref}

\theoremstyle{plain}

\newtheorem{theorem}{Th}
\newtheorem*{theorem_}{Th}
\newtheorem*{statement}{St}
\newtheorem{statement_}{St}
\newtheorem{definition}{Def}
\newtheorem*{definition_}{Def}
\newtheorem{lemma}{Lem}
\newtheorem*{lemma_}{Lem}
\newtheorem*{note}{Nt}
\newtheorem{exersise}{Ex}
\newtheorem{corollary}{Cl}[theorem]
\newtheorem{corollary_}{Cl}[theorem_]

\geometry{top=20mm}
\geometry{bottom=20mm}
\geometry{left=10mm}
\geometry{right=10mm}

\newcommand{\N}{\mathbb N}
\newcommand{\Z}{\mathbb Z}
\newcommand{\Q}{\mathbb Q}
\newcommand{\R}{\mathbb R}
\newcommand{\eps}{\varepsilon}
\renewcommand{\phi}{\varphi}
\renewcommand{\kappa}{\varkappa}

\newcommand{\oR}{\overline{\mathbb R}}
\newcommand{\hR}{\hat{\mathbb R}}

\newcommand{\larrow}{\leftarrow}
\newcommand{\rarrow}{\rightarrow}
\newcommand{\lrarrow}{\leftrightarrow}
\newcommand{\hrarrow}{\hookrightarrow}
\newcommand{\Larrow}{\Leftarrow}
\newcommand{\Rarrow}{\Rightarrow}
\newcommand{\Lrarrow}{\Leftrightarrow}

\hypersetup{
	linktocpage,
    colorlinks=true,
    linktoc=all,
    linkcolor=blue,
}


\begin{document}
	\pagenumbering{gobble}
	\author{Тюленев Александр Иванович\\(Конспектировал Иван-Чай)}
	\date{8 лекция}
	\title{Введение в математический анализ}

	\linespread{1.4}
	\selectfont

	\maketitle
	\newpage

	\tableofcontents

    \begin{note}
        $ \left\{x_n \right\}_{n = 1}^{\infty} $ - числовая последовательность. $ PL(\left\{x_n \right\}_{n = 1}^{\infty} $ - замкнутое множество.
    \end{note}

    \begin{note}
        \[
            \left\{x_n \right\}_{n = 1}^{\infty} \cup PL\left(\left\{x_n \right\}_{n = 1}^{\infty}\right) =
            cl(\left\{x_n \right\}_{n = 1}^{\infty})
        \]
    \end{note}

    \begin{note}
        $ \Q = \left\{P_n \right\}_{n = 1}^{\infty} $ множество т.
    \end{note}

    \section{Предел функции.}

    \begin{definition}
        Под функцией, если не оговорено обратное, понимаем(однозначное) отображение
        $ f: E \to \R $, где $ E \subset \R, E \neq \varnothing $.
    \end{definition}

    \begin{definition}
        Пусть $ \eps > 0, x_0 \in \hR $. Тогда проколотой $ \eps $ окресностью точки
        x_0 называется множество $ U_\eps^.(x_0 := U_\eps (x_0) \setminus { x_0 }.
    \end{definition}

    \begin{note}
        Если $ x_0 = \pm \infty $ или $ x_0 = \infty $ проколотая $ \eps $ окресность
        совпадает с обычной.
    \end{note}

    \begin{definition}[Предела функции по Коши]
        Пусть $ x_0 \in \hR $ и пусть $ A \in \hR. Пусть $ f: U_{\eps}^.(x_0) \to \R $.
        Будем говорить, что A - предел функции f в точке $ x_0 $ и записывать
        $ \lim_{x \to x_0} f(x) = A $ или $ f(x) \to A, x \to x_0 $, если
        \[
            \forall \eps > 0 \quad \exists \delta(\eps) \in (0, \delta_n]:
            \forall x \in U^._\delta(x_0) \hrarrow
            f(x) \in U_{\eps}(A)
        .\]
    \end{definition}

    \begin{definition}
        Псоледовательностью Гейне в точке $ x_0 \in \hR $ называется такая числовая
        последовательность $ \left\{x_n \right\}_{n = 1}^{\infty} \in \R $, что
        \begin{enumerate}
            \item $ \lim_{n \to \infty} x_n = x_0 $
            \item $ x_n \neq x_0 \forall n \in N $
        \end{enumerate}
    \end{definition}

    \begin{definition}[Предела функции по Гейне]
        Пусть $ x_n \in \hR, A \in \hR $. Пусть $ f: U^.\delta(x_n) \to \R $.
        Будем говорить, что $ \exists \lim_{x \to x_0} f = A $ по Гейне, если
        $ \forall последовательноть Гейне $ \left\{x_n \right\}_{n = 1}^{\infty}
        \in U^._\delta(x_0 ) $ в точке $ x_0 $ если $ \exists \lim_{n \to \infty} f(x_n) = A $.
    \end{definition}

    \begin{theorem}[Эквивалентность определений по Коши и по Гейне]
        Пусть $ x_0 \in \hR, A \in \hR $. Пусть $ f: u^._\delta(x_n) \to \R.
        Тогда
        $ \lim_{x \to x_0} f(x) $ (по Коши) $ \Lrarrow $
        $ \lim_{x \to x_0} f(x) $ (по Гейне)
    \end{theorem}

    \begin{proof}[Коши $ \Rarrow $ Гейне]
        Возьмем произвоьную последовательность Гейне $ \{ x_n \} \in U^._\delta(x_n) $
        в точке $ x_0 $.

        \[
            \lim_{n \to \infty} x_n = x_0 \Rarrow \forall \eps > 0 \quad \exists N(\delta) \in \N:
            \forall n \geq N(\delta) \hrarrow x_n \in U_\delta^.(x_n).
        .\]

        В частности, если произвольней $ \eps > 0 $, то
        $ \forall n \geq N(\delta(\eps)) \hrarrow  x_n \in U_\delta(x_n) \Rarrow
        \forall n \geq N(\eps) \hrarrow f(x_n) \in U_{\eps}(A)$
    \end{proof}

    \begin{proof}[Гейне $ \Rarrow $ Коши]
        Пердоположим, что $ \exists \lim_{x \to x_0} f(x) = A $ по Гейне, но не по Коши.

        $ \exists \eps > 0: \forall \delta \in (0, \delta_n] \quad \exists x \in U_\delta^.(x_0):
        f(x) \neq U_{\eps}(A) $

        $ \exists \eps > 0: \forall n \in \N \quad \exists x_n \in U^._{\frac{\delta}{n}}(x_n):
        f(x_n) \not \in U_{\eps}(A) $
    \end{proof}

    \section{Предел по множеству}

    \begin{definition}
        $ E \subset \R $ - Непустое множество, $ x_n \in \hR $. Будем говорить, что $ x_n $ -
        предельная точка множества E, если
        \[
            \forall \delta > 0 \quad U^.\delat(x_0) \cap E \neq \varnothing
        .\].
    \end{definition}

    \begin{definition}[Предела по множесту].
        Пусть A \in \hR, x_0 \in \hR $.
        Пусть $ f : E \to \R, E \neq \varnothing $ и $ x_n $ - предельная точка
        множества E. Будем говорить, что A предел f по множеству E при $ x \to x_0 $,
        если:

        По Коши:
        \[
            \forall \eps > 0 \quad \exists \delta(\eps) > 0: \forall x \in E \cap U^._\delta(x_n) \hrarrow
            f(x) \in U_{\eps}(A)
        .\]

        По Гейне
        $ \forall $ последовательности Гейне $ \{ x_n \} \subset E $ в точке $ x_0 \hrarrow
        \lim_{n \to \infty} f(x_n) = A $
    \end{definition}

    \begin{lemma}
        Пусть $ E_1, E_2 \subset \R $. Пусть $ A \in \hR, x_0 \in \hR $.
        Пусть $ x_0 $ - предельная точка для $ E_1 $ и для $ E_2 $. Тогда в силу утверждения
        эквивалентности

        там крч предел по объеденению Е1 и Е2 тоже самое что два предела по Е1 и по Е2
    \end{lemma}

    \begin{proof}[Слева на право]
        \[
            \forall  \eps > 0 \exists \delta(x) > 0: \forall x
        .\]
    \end{proof}

    \begin{definition}[Функция Дирихле].
        $ f(x) = $
        \left\{
        \begin{aligned}
            0, x \in \R \setminus \Q \\
            1, x \in \Q \\
        \end{aligned}
        \right.
    \end{definition}

    \begin{lemma}
        Пусть $ f: E \to \R, x_0 $ - предельная точка множества E.
        Пусть $ \forall $ последовательности Гейне в точке $ x_0 \quad \exists
        \lim_{n \to \infty} f(x_n) = A \in \hR $
    \end{lemma}

    \begin{proof}

    \end{proof}
    \begin{definition}[Критерий Коши для функции]
        .
    \end{definition}
\end{document}

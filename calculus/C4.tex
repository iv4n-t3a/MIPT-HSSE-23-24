\documentclass[a5paper, 10pt]{article}

\usepackage[english, russian]{babel}
\usepackage[T2A]{fontenc}
\usepackage[utf8]{inputenc}
\usepackage{amsmath, amsfonts, amssymb, amsthm, mathtools}
\usepackage{indentfirst}
\usepackage{soulutf8}
\usepackage{geometry}
\usepackage{ulem}
\usepackage{color}
\usepackage{hyperref}

\theoremstyle{plain}

\newtheorem*{theorem}{Th}
\newtheorem{theorem_}{Th}
\newtheorem*{statement}{St}
\newtheorem{statement_}{St}
\newtheorem{definition}{Def}
\newtheorem*{definition_}{Def}
\newtheorem*{lemma}{Lem}
\newtheorem{lemma_}{Lem}
\newtheorem*{remark}{Rmk}
\newtheorem{exersise}{Ex}
\newtheorem{corollary}{Cl}[theorem]
\newtheorem*{corollary_}{Cl}

\geometry{top=20mm}
\geometry{bottom=20mm}
\geometry{left=10mm}
\geometry{right=10mm}

\newcommand{\N}{\mathbb N}
\newcommand{\Z}{\mathbb Z}
\newcommand{\Q}{\mathbb Q}
\newcommand{\R}{\mathbb R}
\newcommand{\eps}{\varepsilon}
\renewcommand{\phi}{\varphi}
\renewcommand{\kappa}{\varkappa}

\newcommand{\Ro}{\overline{\mathbb R}}
\newcommand{\Rh}{\hat{\mathbb R}}

\newcommand{\larrow}{\leftarrow}
\newcommand{\rarrow}{\rightarrow}
\newcommand{\lrarrow}{\leftrightarrow}
\newcommand{\hrarrow}{\hookrightarrow}
\newcommand{\Larrow}{\Leftarrow}
\newcommand{\Rarrow}{\Rightarrow}
\newcommand{\Lrarrow}{\Leftrightarrow}

\hypersetup{
	linktocpage,
    colorlinks=true,
    linktoc=all,
    linkcolor=blue,
}


\begin{document}
	\pagenumbering{gobble}
	\author{Тюленев Александр Иванович\\(Конспектировал Иван-Чай)}
	\date{13.09.23}
	\title{Введение в математический анализ}

	\linespread{1.4}
	\selectfont

	\maketitle
	\newpage

	\tableofcontents

    \section{Утверждения эквивалентные определению предела.}

    \begin{statement}
        Следущие утверждения эквивалентны $ \forall a \in \Rh, c \geq 1 $

        \begin{enumerate}
            \item $ \lim_{n \to \infty} x_n = a. $
            \item $ \forall \eps > 0 \quad \exists N(\eps): \quad \forall n \geq N(\eps)
                \hrarrow x_n \in U_\eps(a). $
            \item $ \forall \widetilde{\eps} > 0 \quad \exists \widetilde{N} (\widetilde{\eps}):
                \quad \forall n \geq \widetilde{N} (\widetilde{\eps})
                \hrarrow x_n \in U_{\widetilde{\eps} * c}(a). $
            \item $ \forall \eps > 0 \quad \exists N'(\eps) \in \N:
                \forall n > N'(\eps) \hrarrow x_n \in U_{\eps}(a). $
        \end{enumerate}
    \end{statement}

    \begin{proof}[(2) \Rarrow (3):]
        Т.к. $ c \geq 1 $, то $ U_{\eps}(a) \subset U_{c*\eps}(a) $
        \Rarrow при $ \widetilde{N} = N(\eps). $
    \end{proof}

    \begin{proof}[(3) \Rarrow (2):]
        $ \forall \eps > 0 \quad
        \exists N(\eps) := \widetilde{N} (\widetilde{\eps}) =
        \widetilde{N}\left(\frac{\eps}{c}\right):
        \forall n \geq \widetilde{N} \left( \frac{\eps}{c} \right) \hrarrow
        x_n \in U_{\widetilde{\eps} c}(a) = U_{\eps}(a). $
    \end{proof}

    \begin{proof}[(2) \Rarrow (4)]
        $ N'(\eps) = N(\eps). $
    \end{proof}

    \begin{proof}[(4) \Rarrow (2)]
        $ N(\eps) = N'(\eps) + 1. $
    \end{proof}

    \begin{definition}
        Последовательность $ \left\{ x_n \right\}  $ называется сходящейся $ \Lrarrow $
        она имеет конечный предел, в противном случае она называется расходящейся.
    \end{definition}

    \begin{definition}
        Последовательность $ \left\{ x_n \right\}  $ называется ограниченной
        $ \Lrarrow \exists M \in [0, +\infty):
        \quad \forall n \in \N \hrarrow \left| x_n \right| < M. $
    \end{definition}

    \begin{definition}
        Последовательность называется бесконечно большой
        $ \Lrarrow \exists \lim_{n \to \infty} x_n = \infty $.
    \end{definition}

    \begin{statement}
        \left[
        \begin{aligned}
            \lim_{n \to \infty} x_n = + \infty \\
            \lim_{n \to \infty} x_n = - \infty \\
        \end{aligned}
        \right.
        \Rarrow \lim_{n \to \infty} x_n = \infty.
    \end{statement}

    \begin{exersise}
        Как связаны условия

        \begin{enumerate}
            \item $ \left\{ x_n \right\}  $ - неограниченно.
            \item $ \left\{ x_n \right\}  $ - бесконечно большая.
        \end{enumerate}
    \end{exersise}

    \begin{proof}[\neg (1) \Rarrow (2)]
        Контрпример:
        $ \left\{ (1 + (-1)^{n} )*n \right\}_{n = 1}^{\infty}.
    \end{proof}

    \section{Лемма о непересекающихся отрезках}
    \begin{lemma}[Лемма о непересекающихся отрезках.]
        $ \quad \forall a, b \in \Ro, a \neq b \quad \exists \eps > 0:
        U_{\eps}(a) \cap U_{\eps}(b) = \varnothing. $
    \end{lemma}

    \begin{proof}[Для $ -\infty < a < b < +\infty $]
        \[ \eps = \frac{b - a}{2}. \]
    \end{proof}

    \begin{proof}[Для $ -\infty < a < b = +\infty $ ]

        \[ \eps = \frac{1}{\left| a \right| + 1} \leq 1.\]

        $ U_{\eps}(b) = (\left| a \right| + 1, +\infty). $

        $ U_{\eps}(a) = \left( a - \frac{1}{1 + \left| a \right|}, a + \frac{1}{1 + |a|}  \right)
        \subset (a - 1, a + 1). $
    \end{proof}

    \begin{proof}[Для $ -\infty = a < b < +\infty $]
        \[ \eps = \frac{1}{\left| b \right| + 1}. \]
    \end{proof}

    \begin{proof}[Для $ -\infty = a < b = +\infty $]
        \[ \eps = 1. \]
    \end{proof}

    \section{Свойства пределов.}

    \begin{theorem}
        $ \forall \left\{x_n \right\}_{n = 1}^{\infty} \hrarrow
        \exists \lim_{n \to \infty} x_n \in \Ro \Rarrow
        \exists! \lim_{n \to \infty} x_n \in \Ro. $
    \end{theorem}

    \begin{proof}
        Допустим $ \exists a \in \Ro, \exists b \in \Ro: \newline
        a \neq b \land \lim_{n \to \infty} x_n = a \land \lim_{n \to \infty} x_n = b. $

        По предывдущей лемме $ \exists \eps^{*}  > 0: U_{\eps^{*}}(a) \cap U_{\eps^{*} }(b)
        = \varnothing $

        \[ \lim_{n \to \infty} x_n = a \Lrarrow \forall \eps > 0 \exists N_1(\eps) \in \N:
        \forall n \geq N_1(\eps) \hrarrow x_n \in U_{\eps}(a). \]
        \[ \lim_{n \to \infty} x_n = b \Lrarrow \forall \eps > 0 \exists N_2(\eps) \in \N:
        \forall n \geq N_2(\eps) \hrarrow x_n \in U_{\eps}(a). \]

        $ n > max \left\{ N_1, N_2 \right\} $, то $ x_n \in U_{\eps*}(a) \cap U_{\eps*}(b) =
        \varnothing $ - противоречие.
    \end{proof}

    \begin{remark}
        В $ \Ro $ предел не единственен.
    \end{remark}

    \begin{theorem}
        Если последовательность $ \left\{x_n \right\}_{n = 1}^{\infty} $ - сходится, то
        она ограниченна.
    \end{theorem}

    \begin{proof}
        Пусть $ \lim_{n \to \infty} x_n = a \in \R \Rarrow
        \forall \eps > 0 \quad \exists N(\eps) \in \N: \quad \forall n \geq \N \hrarrow
        x_n \in U_{\eps}(a). $

        В частности $ \exists N = N(1): \forall n \geq N(1) \hrarrow |x_n| \leq |a| + 1 $

        $ M = max \left\{ |x_1, x_2, \dots |x_{N(1)}| \right\} \Rarrow
        \left| x_n \right| \leq \left| a \right| + 1 \quad \forall n \in \N. $
    \end{proof}

    \begin{remark}
        Обратное не верно.
    \end{remark}

    \begin{proof}
        Контрпример $ \left\{ (-1)^{n} \right\}_{n = 1}^{\infty}. $
    \end{proof}

    \begin{definition}
        Число $ M $ называется максимальным элементом множества $ E \subset \R
        \Lrarrow M = \max E $, если

        \begin{enumerate}
            \item $ M \in E. $
            \item $ M \geq x \quad \forall x \in E. $
        \end{enumerate}
    \end{definition}

    \begin{exersise}
        Доказать, что $ \sup{A} \in A \Lrarrow \sup{A} = \max{A}. $
    \end{exersise}

    \section{Свойства пределов сходящихся последовательности, связаные с арифметическими
        операциями}

    \begin{definition}
        Последовательность $ \left\{ x_n \right\} $ называется бесконечно малой,
        если $ \exists \lim_{n \to \infty} = 0. $
    \end{definition}

    \begin{lemma}
        Если $ \left\{ x_n \right\}  $ - ограничена, $ \left\{ y_n \right\} $ - бесконечно малая,
        то $ \left\{ z_n \right\} := \left\{ x_n y_n \right\}_{n = 1}^{\infty} $ - бесконечно
        малая.
    \end{lemma}

    \begin{proof}
        Запишем определения

        \begin{cases}
            $ x_n $ - ограничена \\
            $ y_n $ - бесконечно малая \\
        \end{cases}

        \begin{cases}
        $ \Lrarrow \exists M \geq 0: |x_n| \leq M \quad \forall n \in \N $ \\
        $ \Lrarrow \forall \eps > 0 \quad \exists N \in N:
            \forall n \geq N \hrarrow |y_n| < \eps $
        \end{cases}

        \[ \forall \eps > 0 \quad \exists N(\eps) \in \N: \forall n \geq N(\eps) \Rarrow
        |y_n x_n| < M * \eps.\]
        (В силу утверждения из начала конспекта.)
        \[\lim_{n \to \infty} z_n = 0. \]
    \end{proof}

    \begin{lemma}
        Сумма, разность и произведение бесконечно малых последовательностей есть
        бесконечно малая последовательность.

        Если $ \left\{x_n \right\}_{n = 1}^{\infty} $ и
        $ \left\{x_n \right\}_{n = 1}^{\infty} $ - бм, то
        $ \left\{ x_n \pm y_n \right\} $ и
        $ \left\{ x_n y_n \right\} $ - бм.
    \end{lemma}

    \begin{proof}[Для суммы и разности]
        :

        \begin{cases}
            \forall \eps > 0 \quad \exists N_1 \in \N: \forall n \geq N_1 \hrarrow
            x_n \in U_{\frac{\eps}{2}}(0) \\
            \forall \eps > 0 \quad \exists N_2 \in \N: \forall n \geq N_2 \hrarrow
            x_n \in U_{\frac{\eps}{2}}(0) \\
        \end{cases}
        $ \Rarrow $
        \newline
        $ \forall \eps > 0 \quad \exists N(\eps) := \max \left\{ N_1, N_2 \right\}:
        \forall n \geq  N(\eps) \hrarrow x_n \pm y_n \in U_{\eps}(0). $
    \end{proof}

    \begin{proof}[Для произведения]
        $ \left\{ x_n y_n \right\} $ - бм, т.к. $ \left\{ x_n \right\} $ - ограничена,
        $ \left\{ y_n \right\} $ - бм.
    \end{proof}

    \begin{lemma}
    $ \lim_{n \to \infty} x_n = a \in \R \Lrarrow \left\{ x_n - a \right\} $ - бм.
    \end{lemma}

    \begin{corollary_}
        Пусть $ \lim_{n \to \infty} a_n = a \in \R $ и $ \lim_{n \to \infty} b_n = b \in \R $,
        тогда:

        \[ \lim_{n \to \infty} (a_n \pm b_n) = a \pm b. \]
        \[ \lim_{n \to \infty} (a_n b_n) = a b. \]
    \end{corollary_}

    \begin{proof}[Для суммы и разности]
        $ \left\{ (a_n \pm b_n) - (a \pm b) \right\}_{n = 1}^{\infty} $ - бм.
    \end{proof}

    \begin{proof}[Для произведения]
        Покажем, что
        $ \left\{ (a_n b_n) - (a b) \right\}_{n = 1}^{\infty} $ - бм.

        $ a_n b_n - ab = a_n b_n - a_n b + a_n b - ab = a_n (b_n - b) + b (a_n - a). $

        Рассмотрим последовательности

        $ \left\{ a_n (b_n - b) \right\}_{n = 1}^{\infty} $ - бм, как произведение
        ограниченной $ \{ a_n \} $ на бм $ \{ b_n - b \}.

        $ \left\{ b (a_n - a) \right\}_{n = 1}^{\infty} $ - бм, как произведение
        стационарной $ x_n = b $ на бм $ \{ b_n - b \}.
    \end{proof}

    \begin{lemma}
        Пусть $ x_n \neq 0 \quad \forall n \in \N $ и $ \exists \lim_{n \to \infty} x_n = x \in \R,
        x \neq 0 $, тогда $ \exists \lim_{n \to \infty} \frac{1}{x_n} = \frac{1}{x}. $
    \end{lemma}

    \begin{proof}
        Покажем, что $ \left\{ x_n \right\}  $ - ограничена.

        \[ \forall \eps > 0 \quad \exists N(\eps): \forall n > N \hrarrow x_n \in U_{\eps}(x). \]

        В частности для $ \eps = \frac{|x|}{2}, то \quad \exists N^* \in \N: \forall n \geq N^*
        \hrarrow x_n \in U_{\eps}(x) \Rarrow x - \frac{|x|}{2} < x_n < x + \frac{|x|}{2} $
        $ \Rarrow n \geq N^* \hrarrow |x_n| \geq \frac{|x|}{2} $
        $ \Rarrow n \geq N^* \hrarrow \frac{1}{|x_n|} \leq \frac{2}{|x|}.$

        $ M := \left\{
            \frac{1}{|x_1|},
            \frac{1}{|x_2|},
            \frac{1}{|x_3|},
            \dots
            \frac{1}{|x_{N^*}|},
            \frac{2}{|x|},
        \right\}  $, тогда
        $ \frac{1}{|x_n|} \leq M \quad \forall n \in \N \Rarrow \left\{ \frac{1}{x_n}  \right\} $
        - ограничена.

        \[ \frac{1}{x_n} - \frac{1}{x} = \frac{x - x_n}{x x_n} = \frac{1}{x x_n} (x - x_n). \]

            $ \left\{ x - x_n \right\} $ - бм,
            $ \left\{ \frac{1}{x x_n} \right\} $ - бм
            $ \Rarrow \left\{ \frac{1}{x_n} - \frac{1}{x} \right\} $ - бм
            $ \Rarrow \lim_{n \to \infty} \frac{1}{x_n} = \frac{1}{x}.  $
    \end{proof}

    \begin{corollary_}
        Пусть
        $ \exists \lim_{n \to \infty} y_n = y \in \R $,
        $ \exists \lim_{n \to \infty} x_n = x \in \R $.
        Пусть $ x \neq 0 \land x_n \neq 0 \quad \forall n \in \N $.
        Тогда \[ \exists \lim_{n \to \infty} \frac{y_n}{x_n} = \frac{y}{x}. \]
    \end{corollary_}
\end{document}

% /*
%  * ----------------------------------------------------------------------------
%  * "THE BEER-WARE LICENSE" (Revision 42):
%  * @iv4n-t3a wrote this file.  As long as you retain this notice you
%  * can do whatever you want with this stuff. If we meet some day, and you think
%  * this stuff is worth it, you can buy me a beer in return.   Иван-Чай
%  * ----------------------------------------------------------------------------
%  */

\documentclass[a5paper, 10pt]{article}

\usepackage[english, russian]{babel}
\usepackage[T2A]{fontenc}
\usepackage[utf8]{inputenc}
\usepackage{amsmath, amsfonts, amssymb, amsthm, mathtools}
\usepackage{indentfirst}
\usepackage{soulutf8}
\usepackage{geometry}
\usepackage{ulem}
\usepackage{color}
\usepackage{hyperref}

\theoremstyle{plain}

\newtheorem{theorem}{Th}
\newtheorem*{theorem_}{Th}
\newtheorem*{statement}{St}
\newtheorem{statement_}{St}
\newtheorem{definition}{Def}
\newtheorem*{definition_}{Def}
\newtheorem{lemma}{Lem}
\newtheorem*{lemma_}{Lem}
\newtheorem*{note}{Nt}
\newtheorem{exersise}{Ex}
\newtheorem{corollary}{Cl}[theorem]
\newtheorem{corollary_}{Cl}[theorem_]

\geometry{top=20mm}
\geometry{bottom=20mm}
\geometry{left=10mm}
\geometry{right=10mm}

\newcommand{\N}{\mathbb N}
\newcommand{\Z}{\mathbb Z}
\newcommand{\Q}{\mathbb Q}
\newcommand{\R}{\mathbb R}
\newcommand{\eps}{\varepsilon}
\renewcommand{\phi}{\varphi}
\renewcommand{\kappa}{\varkappa}
\renewcommand{\vec}{\overrightarrow}

\newcommand{\oR}{\overline{\mathbb R}}
\newcommand{\hR}{\hat{\mathbb R}}

\newcommand{\larrow}{\leftarrow}
\newcommand{\rarrow}{\rightarrow}
\newcommand{\lrarrow}{\leftrightarrow}
\newcommand{\hrarrow}{\hookrightarrow}
\newcommand{\Larrow}{\Leftarrow}
\newcommand{\Rarrow}{\Rightarrow}
\newcommand{\Lrarrow}{\Leftrightarrow}

\hypersetup{
	linktocpage,
    colorlinks=true,
    linktoc=all,
    linkcolor=blue,
}


\begin{document}
	\pagenumbering{gobble}
	\author{Тюленев Александр Иванович\\(Конспектировал Иван-Чай)}
	\date{9 лекция}
	\title{Введение в математический анализ}

	\linespread{1.4}
	\selectfont

	\maketitle
	\newpage

	\tableofcontents

    \section{Дисклеймер}

    Эта лекция еще в разработки, так-что она не содержит всей информации из лекции + \newline
    ГДЕ-ТО(ПОЧТИ ВЕЗДЕ) ПРЕДЕЛ ПО МНОЖЕСТВУ ЗАПИСАН КАК ОБЫЧНЫЙ

    \section{Условие Коши}


    \begin{theorem}[Условие Коши]
        Пусть $ f: \dot{U}_{\delta_0} \to \R $.
        \[
            \forall \eps > 0 \exists \delta(\eps) \in (0, \delta_0]: \forall x_1, x_2 \in \dot{U}_{\delta(\eps)}(x_0) \hrarrow |f(x_1) - f(x_2)| < \eps |
        .\]
    \end{theorem}
    \begin{theorem_}[Критерий Коши]
        Пусть $ f: \dot{U}_{\delta_0}(x_0) \to \R $.
        Следущие условия эквивалентны.
        \begin{enumerate}
            \item f удовлетворяет усовию Коши в $ x_0 $
            \item $ \exists \lim\limits_{x \to x_0} = a \in \R $
        \end{enumerate}
    \end{theorem_}

    \begin{proof}[(1) $ \Rarrow $ (2)]
        Пусть $ \exists \lim\limits_{x \to x_0} f(x) = a \in \R $.
        \[ \forall \eps > 0 \quad \exists \delta \in (0, \delta_0]: \quad \forall x \in \dot{U}(x_0) \hrarrow |f(x) - a| < \frac{\eps}{2} \]
        \[ \Downarrow \]
        \[ \forall \eps > 0 \quad \exists \delta \in (0, \delta_0]: \quad \forall x_1, x_2 \in \dot{U}(x_0) \hrarrow
            |f(x_1) - a| < \frac{\eps}{2} \land
            |f(x_2) - a| < \frac{\eps}{2} \]
    \end{proof}

    \begin{proof}[(2) $ \Rarrow $ (1)]
        Поскольку определения по Коги и по Гейне эквивалентны докажем,что из критерия Коши следует существование предела
        по Гейне.

        Зафиксируем произвольную последовательность Гейне в точке $ x_0 $.

        \[
            \forall \delta > 0 \quad \exists N(\delta) \in \N: \quad \forall n \geq N \hrarrow  x_n \in \dot{U}_{\delta}(x_0)
        .\] \[ \Downarrow \] \[
            \forall \delta > 0 \quad \exists N(\delta(\eps)) \in \N: \quad \forall n, m \geq N(\delta(\eps)) \hrarrow |f(x_n) - f(x_m)| < \eps
        .\]

        Получилось условие Коши для $ \left\{ f(x_n) \right\}_{n = 1}^{\infty} $, отсюда
        $ \exists \lim\limits_{n \to \infty} f(x_n) = A \in \R $.

        Для произвольной последовательности Гейне в точке $ x_0 $
        \[
            \exists \lim\limits_{n \to \infty} f(x_n) = A
        .\]
        Но в силу предыдущей леммы A не зависит от выбора $ x_n $.
    \end{proof}

    \begin{note}
        Критерий Коши работает и для пределов по множеству.
    \end{note}

    \begin{proof}
        Доказательсвто аналогично предыдущему.
    \end{proof}

    \begin{theorem_}[Принцип локализации]
        Пусть
        \[\exists \overline{\delta} \in (0, +\infty): f(x) = g(x) \forall x \in \dot{U}_{\overline{\delta}}(x_0). \]
        Тогда

    \end{theorem_}

    \section{Односторонний предел и т. Вейерштрассе}

    \begin{definition}
        Пусть $ f: \dot{U}^+_{\delta_0}(x_0) \to \R, \quad \dot{U}^+_{\delta_0}(x_0) = (x_0, x_0+\delta) $.
        Будем говорить, что $ A \in \hR $ является правостороним пределом $ f $ в $ x_0 $, если
        \[
            \lim\limits_{{x \to x_0}\limits_{x \in \dot{U}^+_{\delta_0}(x_0)} = A
        .\]
    \end{definition}

    \begin{definition}
        Функция называется нестрого возрастающей(убывающей) на $ x \subset \R, X \neq \varnothing $, если
        $ \forall x_1, x_2 \in X \quad x_1 < x_2 \Rarrow f(x_1) \leq f(x_2) \quad (f(x_1) \geq f(x_2)) $.
    \end{definition}

    Аналогично определяется строгое возрастание/убываение на X, строгая/нестрогая монотонность на X.

    \begin{definition}
        \[
            \sup_{x \in X} f(x) = \sup \{ f(x), x \in X \}.
        \] \[
            \inf_{x \in X} f(x) = \inf \{ f(x), x \in X \}.
        \] \[
            \max_{x \in X} f(x) = \max \{ f(x), x \in X \}.
        \] \[
            \min_{x \in X} f(x) = \min \{ f(x), x \in X \}.
        \]
    \end{definition}

    Запишем в кванторах \newline
    $ \sup\limits_{x \in X} f(x) = M \in \R \Lrarrow $
    \begin{cases}
        f(x) \leq M, \quad \forall  x \in X \\
        \forall M' < M \quad \exists x' \in X \quad f_(x') > M' \\
    \end{cases}

    $ \sup\limits_{x \in X} f(x) = M \in \R \Lrarrow $
    \begin{cases}
        f(x) \leq M, \quad \forall  x \in X \\
        \forall \eps > 0 \quad ex x_\eps \in X: f(x_\eps) \in U_{\eps}(M)
    \end{cases}

    \begin{theorem_}[Теорема Вейерштрассе]
        Пусть $ -\infty \leq a < b \leq +\infty $.
        Пусть f нестрого возрастает на $ (a, b) $.
        Тогда
        \[
            \exists \lim\limits_{x \to x_0 - 0} = \sup\limits_{x \in (a, b)} f(x)
        \] \[
            \exists \lim\limits_{x \to x_0 + 0} = \inf\limits_{x \in (a, b)} f(x)
        .\]
    \end{theorem_}

    \begin{proof}[Докажем (1) т.к. (2) аналогично]
        \[
            E = \{ f(x): x \in (a, b) \}
        .\] \[
            \exists \sup E = M \Rarrow \forall \eps > 0 \quad \exists x_\eps \in (a, b): \quad f(x_\eps) \in U_{\eps}(M)
        .\]
        Но f нестрогов возрастает на $ (a, b) \Rarrow f(x) \in U_{\eps}(M) \quad \forall x \in [x_\eps, b) $

        Но тогда $ \forall  \eps > 0 \quad \exists \delta(\eps) = b - x_\eps: \forall x \in \dot{U}^{-}_{\delta(\eps)}(b) \hrarrow f(x) \in U_{\eps}(M) $
    \end{proof}

    \section{Арифмитические операции с пределами функции}

    \begin{theorem}
        Пусть $ x \neq \varnothing, x_0 $ - предельная точка.
        Пусть $ \exists \lim\limits_{x \to x_0} f_1(x) = a_1 \in \R $
        Пусть $ \exists \lim\limits_{x \to x_0} f_2(x) = a_2 \in \R $
        Тогда
        \begin{enumerate}
            \item $ \lim\limits_{x \to x_0} (f_1(x) \pm f_2(x)) = a_1 \pm a_2 $.
            \item $ \lim\limits_{x \to x_0} (f_1(x) \cdot f_2(x)) = a_1 \cdot a_2 $.
        \end{enumerate}
    \end{theorem}

    \begin{proof}
        Для доказательства следует использовать определение предела по Гейне, т.к. для
        последовательностей подобные свойства уже доказаны.
    \end{proof}

    \begin{lemma}[Лемма о сохранение знаков]
        Пусть $ f: x \to \R $, $ x_0 $ - предельная точка X.
        Пусть $ \exists \lim\limits_{x \to x_0} f(x) = a \neq 0 $.
        Тогда $ \exists \overline{\delta} > 0: \forall x \in \dot{U}_{\overline{\delta}}(x_0) \cap X \hrarrow $
        \begin{cases}
            f(x) \neq 0 \\
            sign f(x) = sign a \\
        \end{cases}
    \end{lemma}

    \begin{corollary}
        Пусть $ f, g: X \to \R, X \neq \varnothing, x_0 $ - предельня точка X.
        Пусть $ \exists \lim\limits_{x \to x_0} g(x) = B \neq 0. $
        Тогда $ \exists \overline{\delta} > 0: g(x) \neq 0 \quad \forall x \in \dot{U}_{\overline{\delta}}(x_0) \cap X $
    \end{corollary}

    \begin{proof}
        Следует из леммы о сохранение знаков и о пределе частичного для последовательности.
    \end{proof}

    \section{О предельном переходе в неравенство}

    \begin{theorem_}[О трех милиционерах]
    Пусть $ f, g, h: X \to \R, x_0 $ - предельная точка для $ X $.
    Пусть $ \exists \lim\limits_{x \to x_0} f(x) = \lim\limits_{x \to x_0} g(x) \in \oR $.
    Тогда $ \exists \lim\limits_{x \to x_0} h(x) = A.
    \end{theorem_}

    \begin{proof}
        По Гейне.
    \end{proof}

    \begin{theorem}[О предельном переходе в неравенство]
        Пусть $ f, g: X \to \R $, $ x_0 $ - предельная точка для X.
        Пусть
        \begin{cases}
            \exists \lim\limits_{x \to x_0} f(x) = A \in \oR \\
            \exists \lim\limits_{x \to x_0} g(x) = B \in \oR \\
        \end{cases}
        тогда, если $ f(x) \leq g(x) \quad \forall x \in X \Rarrow A \leq B $.
    \end{theorem}

    \begin{note}
        Строгое неравенство может не сохраниться при предельном переходе.
    \end{note}

    \section{Нижний и верхний предел последовательности}

    \begin{definition}
        Пусть $ f : X \to \R, x_0 $ - предельная точка.
    \end{definition}
    
    
\end{document}

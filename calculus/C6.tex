% /*
%  * ----------------------------------------------------------------------------
%  * "THE BEER-WARE LICENSE" (Revision 42):
%  * @iv4n-t3a wrote this file.  As long as you retain this notice you
%  * can do whatever you want with this stuff. If we meet some day, and you think
%  * this stuff is worth it, you can buy me a beer in return.   Иван-Чай
%  * ----------------------------------------------------------------------------
%  */

\documentclass[a5paper, 10pt]{article}

\usepackage[english, russian]{babel}
\usepackage[T2A]{fontenc}
\usepackage[utf8]{inputenc}
\usepackage{amsmath, amsfonts, amssymb, amsthm, mathtools}
\usepackage{indentfirst}
\usepackage{soulutf8}
\usepackage{geometry}
\usepackage{ulem}
\usepackage{color}
\usepackage{hyperref}

\theoremstyle{plain}

\newtheorem{theorem}{Th}
\newtheorem*{theorem_}{Th}
\newtheorem*{statement}{St}
\newtheorem{statement_}{St}
\newtheorem{definition}{Def}
\newtheorem*{definition_}{Def}
\newtheorem{lemma}{Lem}
\newtheorem*{lemma_}{Lem}
\newtheorem*{note}{Nt}
\newtheorem{exersise}{Ex}
\newtheorem{corollary}{Cl}[theorem]
\newtheorem{corollary_}{Cl}[theorem_]

\geometry{top=20mm}
\geometry{bottom=20mm}
\geometry{left=10mm}
\geometry{right=10mm}

\newcommand{\N}{\mathbb N}
\newcommand{\Z}{\mathbb Z}
\newcommand{\Q}{\mathbb Q}
\newcommand{\R}{\mathbb R}
\newcommand{\eps}{\varepsilon}
\renewcommand{\phi}{\varphi}
\renewcommand{\kappa}{\varkappa}

\newcommand{\oR}{\overline{\mathbb R}}
\newcommand{\hR}{\hat{\mathbb R}}

\newcommand{\larrow}{\leftarrow}
\newcommand{\rarrow}{\rightarrow}
\newcommand{\lrarrow}{\leftrightarrow}
\newcommand{\hrarrow}{\hookrightarrow}
\newcommand{\Larrow}{\Leftarrow}
\newcommand{\Rarrow}{\Rightarrow}
\newcommand{\Lrarrow}{\Leftrightarrow}

\hypersetup{
	linktocpage,
    colorlinks=true,
    linktoc=all,
    linkcolor=blue,
}


\begin{document}
	\pagenumbering{gobble}
	\author{Тюленев Александр Иванович\\(Конспектировал Иван-Чай)}
	\date{6 лекция}
	\title{Введение в математический анализ}

	\linespread{1.4}
	\selectfont

	\maketitle
	\newpage

	\tableofcontents

    \section{Частичные пределы}

    \begin{theorem}
        Если $ \left\{x_n \right\}_{n = 1}^{\infty} $ - неограничена сверху, то
        $ + \infty $ является ее частичным пределом.

        Если $ \left\{x_n \right\}_{n = 1}^{\infty} $ - неограничена снизу, то
        $ - \infty $ является ее частичным пределом.
    \end{theorem}

    \begin{proof}[Докажем для случая ограниченности сверху]
        Заметим, что если $ \left\{x_n \right\}_{n = 1}^{\infty} $ неограничена сверху, то
        $ \forall n \in \N $ отбросим первые $ N $ элементов и снова получим последовательность
        неограниченную сверху. Т.е. рассмотрим $ \left\{y_n \right\}_{n = 1}^{\infty} =
        \left\{ x_{n + N} \right\}_{n = 1}^{\infty} $.
        \[
            \forall \eps > 0 \quad \exists n \in \N: y_n > \frac{1}{\eps}
        .\] \[ \Downarrow \] \[
            \forall N \in \N, \forall \eps > 0 \quad \exists k > N: x_k > \frac{1}{eps}
        .\]
        $ +\infty $ - частичный предел $ \left\{x_n \right\}_{n = 1}^{\infty} $ по критерию частичного предела.
    \end{proof}

    \begin{theorem_}[Обобщенная теорема Больцаро-Вейерштрасса]
        Любая числовая последовательность $ \left\{x_n \right\}_{n = 1}^{\infty} $
        имеет хотя бы один частичный предел в $ \oR $.
    \end{theorem_}

    \begin{definition}
        $ PL\left( \left\{x_n \right\}_{n = 1}^{\infty} \right) := \{ L \in \oR: L - $ частичный предел $ \} $
    \end{definition}

    \begin{definition}
        Пусть $ A \subset \oR $.
        $ M = \sup A \Lrarrow $
        \begin{cases}
            a \leq M, \forall a \in A \\
            \forall M' < M \quad \exists x \in A: M' < a \leq M.
        \end{cases}
    \end{definition}

    \begin{definition}
        Пусть $ A \subset \oR $.
        $ m = \sup A \Lrarrow $
        \begin{cases}
            a \geq m, \forall a \in A \\
            \forall m' > m \quad \exists x \in A: m \leq a < m'.
        \end{cases}
    \end{definition}

    \section{Верхний и нижний частичный предел}

    \begin{definition}
        Верхний предел \[ \overline{\lim}_{n \to \infty} x_n = \sup PL(\left\{x_n \right\}_{n = 1}^{\infty}). \]
    \end{definition}

    \begin{definition}
        Нижний предел \[ \underline{\lim}_{n \to \infty} x_n = \inf PL(\left\{x_n \right\}_{n = 1}^{\infty}). \]
    \end{definition}

    \begin{lemma}
        Пусть $ \exists \lim_{n \to \infty} x_n = A \in \oR $. Тогда $ PL(x_n) = {A} $.
    \end{lemma}

    \begin{proof}
        \[
            \forall \eps > 0 \quad \exists N(\eps) \in \N: \forall n \geq N(\eps) \hrarrow x_n \in U_{\eps}(A)
        .\]

        Возьмем произвольную подпоследовательность $  \left\{ x_{n_k} \right\}_{k = 1}^{\infty} $ последовательности
        $ \left\{x_n \right\}_{n = 1}^{\infty} $ и покажем, что $ \lim_{k \to \infty} x_{n_k} = A $.

        \[
            n_k \geq k \Rarrow \forall \eps > 0 \quad \exists K(\eps) = N(\eps): \forall k \geq K(\eps) \hrarrow x_{n_k} \in U_{\eps}(A)
        .\]
    \end{proof}

    \begin{theorem}
        Пусть $ \left\{x_n \right\}_{n = 1}^{\infty} $ - произвольная числовая последовательность, тогда
        \[
            \underline{\lim}_{n \to \infty} \in PL(\left\{x_n \right\}_{n = 1}^{\infty})
        .\] \[
            \overline{\lim}_{n \to \infty} \in PL(\left\{x_n \right\}_{n = 1}^{\infty})
        .\]
    \end{theorem}

    \begin{proof}[Докажем для верхнего предела]
        Обозначим $ M = \sup PL(\left\{x_n \right\}_{n = 1}^{\infty}) $

        Из определения $ \sup $
        \[
            \forall \eps > 0 \hrarrow U_{\frac{\eps}{2}} \cap PL(\left\{x_n \right\}_{n = 1}^{\infty}) \neq \varnothing
        .\] \[ \Downarrow \] \[
            \exists c \in PL(\left\{x_n \right\}_{n = 1}^{\infty}): c \in U_{\frac{\eps}{2}}(M)
        .\]

        По критерию частичного предела в $ U_{\frac{\eps}{2}}(c) $ содержится значения бесконечного количества элементов
        $ \left\{x_n \right\}_{n = 1}^{\infty} $.

        \[
            c \in U_{\frac{\eps}{2}} \Rarrow U_{\frac{\eps}{2}} \subset U_{\eps}(M)
        .\]

        Из этого $ U_{\eps}(M) $ содержит значения бесконечного количества значений элементов
        $ \left\{x_n \right\}_{n = 1}^{\infty} $, но $ \eps $ был выбран произвольно. $ \Rarrow $
        $ M $ - частичный предел.
    \end{proof}

    \begin{theorem}
        Пусть $ \left\{x_n \right\}_{n = 1}^{\infty} $ - числовая последовательность, тогда

        \[
            \overline{\lim}_{n \to \infty} x_n = \inf_{n \in \N}\left(\sup_{k \geq n} x_k\right)
        .\] \[
            \underline{\lim}_{n \to \infty} x_n = \sup_{n \in \N}\left(\inf_{k \geq n} x_k\right)
        .\]
    \end{theorem}

    \begin{proof}[Докажем для верхнего предела]
        \[
            \forall n \in \N \quad y_n = \sup_{k \geq n} x_k
        .\]

        Заметим, что $ y_{n+1} \leq y_n \quad \forall n \in \N $

        Если $ y_n = + \infty $ хотя бы при одном $ n \in \N $, то $ y_n = +\infty \quad \forall n \in \N $.
        Получается монотонная последовательность, которая по теореме Вейерштрасса имеет предел равный ее $ \inf $.

        Докажем, что $ \displaystyle \overline{\lim}_{n \to \infty} x_n \leq \inf_{n \in \N} \sup_{k \geq n} x_n $,
        для этого докажем, что $ \displaystyle c \leq \inf_{n \in \N} \sup_{k \geq n} x_n $, если c - частичный предел.

        Пусть $ \left\{ x_{n_k} \right\}_{k = 1}^{\infty} $ - подпоследовательность последовательности
        $  \left\{x_n \right\}_{n = 1}^{\infty}: \newline
        \lim\limits_{k \to \infty} x_{n_k} = c $

        Если $ y_n \in \R \quad \forall n \in \N
    \end{proof}

    \textbf{Дальнейшая часть лекции пока в разработке, довольствуемся тем, что имеем}
\end{document}

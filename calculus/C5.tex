\documentclass[a5paper, 10pt]{article}

\usepackage[english, russian]{babel}
\usepackage[T2A]{fontenc}
\usepackage[utf8]{inputenc}
\usepackage{amsmath, amsfonts, amssymb, amsthm, mathtools}
\usepackage{indentfirst}
\usepackage{soulutf8}
\usepackage{geometry}
\usepackage{ulem}
\usepackage{color}
\usepackage{hyperref}

\theoremstyle{plain}

\newtheorem{theorem}{Th}
\newtheorem*{theorem_}{Th}
\newtheorem*{statement}{St}
\newtheorem{statement_}{St}
\newtheorem{definition}{Def}
\newtheorem*{definition_}{Def}
\newtheorem{lemma}{Lem}
\newtheorem*{lemma_}{Lem}
\newtheorem*{note}{Nt}
\newtheorem{exersise}{Ex}
\newtheorem{corollary}{Cl}[theorem]
\newtheorem{corollary_}{Cl}[theorem_]

\geometry{top=20mm}
\geometry{bottom=20mm}
\geometry{left=10mm}
\geometry{right=10mm}

\newcommand{\N}{\mathbb N}
\newcommand{\Z}{\mathbb Z}
\newcommand{\Q}{\mathbb Q}
\newcommand{\R}{\mathbb R}
\newcommand{\eps}{\varepsilon}
\renewcommand{\phi}{\varphi}
\renewcommand{\kappa}{\varkappa}

\newcommand{\oR}{\overline{\mathbb R}}
\newcommand{\hR}{\hat{\mathbb R}}

\newcommand{\larrow}{\leftarrow}
\newcommand{\rarrow}{\rightarrow}
\newcommand{\lrarrow}{\leftrightarrow}
\newcommand{\hrarrow}{\hookrightarrow}
\newcommand{\Larrow}{\Leftarrow}
\newcommand{\Rarrow}{\Rightarrow}
\newcommand{\Lrarrow}{\Leftrightarrow}

\hypersetup{
	linktocpage,
    colorlinks=true,
    linktoc=all,
    linkcolor=blue,
}


\begin{document}
	\pagenumbering{gobble}
	\author{Тюленев Александр Иванович\\(Конспектировал Иван-Чай)}
	\date{5 лекция}
	\title{Введение в математический анализ}

	\linespread{1.4}
	\selectfont

	\maketitle
	\newpage

	\tableofcontents

    \section{Предельный переход в неравенство}

    \begin{lemma}
        Пусть $ A, B \in \oR $. Пусть $ \lim_{n \to \infty} x_n = A $,
        $ \lim_{n \to \infty} y_n = B $, $ A < B $

        Тогда $ \exists N \in \N: \quad \forall n \geq N \hrarrow x_n < y_n $
        % Тогда $ \exists \eps^* > 0: U_{\eps^*}(A) \cap U_{\eps^*}(B) \neq \varnothing $
    \end{lemma}

    \begin{proof}
        \[
        \exists \eps^* > 0: U_{\eps^*}(A) \cap U_{\eps^*}(B) = \varnothing
        .\] \[
        A < B \Rarrow
        \forall x \in U_{\eps^*}(A), \forall y \in U_{\eps^*}(B) \hrarrow x < y.
        .\] \[
        \forall \eps > 0 \quad \exists N_1(\eps) \in \N
        \quad \forall n \geq N_1 \hrarrow x_n \in U_{\eps}(A)
        .\] \[
        \forall \eps > 0 \quad \exists N_2(\eps) \in \N
        \quad \forall n \geq N_2 \hrarrow y_n \in U_{\eps}(B)
        .\] \[
        N := max\{N_1(\eps^*), N_2(\eps^*) \}: \forall n \geq N \hrarrow x_n
        \in U_{\eps^*}(A) \land
        y_n \in  U_{\eps^*}(B) \Rarrow
        .\] \[
        x_n < y_n
        .\]
    \end{proof}

    \begin{theorem}[Теорема о предельном переходе в неравенство]
        Пусть
        \begin{cases}
            \exists \lim_{n \to \infty} x_n = A \in \oR \\
            \exists \lim_{n \to \infty} y_n = B \in \oR \\
        \end{cases},
        пусть $ \exists N \in \N: x_n \leq y_n \quad \forall n \geq N $,
        Тогда $ A \geq B $
    \end{theorem}

    \begin{proof}
        Предположим, что $ A > B $. По только что доказаной лемме
        $ \exists N^*: \forall n \geq N^* \hrarrow x_n > y_n $
        \[
            \widetilde{N} := max\{N, N^*\}: \quad \forall n \geq  \widetilde{N} \hrarrow
            \begin{cases}
                x_n \leq y_n \\
                x_n > y_n
            \end{cases}
        \]
        Противоречие.
    \end{proof}

    \begin{statement}
        Пусть $ \exists N \in \N: x_n < y_n \quad \forall n \geq N $, \newline
        $ x_n \to A, n \to \infty $, \newline
        $ y_n \to B, n \to \infty $. \newline
        Тогда не обязательно $ A < B $.
    \end{statement}

    \begin{proof}[Контрпример]
        \[
            y_n = \frac{1}{n}
        .\] \[
            x_n = -\frac{1}{n}
        .\]
    \end{proof}

    \begin{note}
        Предельный переход может портить строгие неравенства и превращать их в нестрогие.
    \end{note}

    \begin{corollary}
        \left.
        \begin{aligned}
            x_n \geq a, a \in \R \\
            \exists \lim_{n \to \infty} x_n = A \in \oR \\
        \end{aligned}
        \right\}
        \Rarrow A \geq a.
    \end{corollary}

    \begin{proof}
        Положим $ y_n = a $. $ \forall n \in \N $
        применим предыдущее утверждение(теорему о предельном переходе в неравенство).
    \end{proof}

    \section{Теорема о двух милиционерах}

    \begin{theorem_}[Теорема о двух милиционерах $ \Lrarrow $ теорема о трех последовательностях]
        Пусть $
        \left\{a_n \right\}_{n = 1}^{\infty},
        \left\{b_n \right\}_{n = 1}^{\infty},
        \left\{c_n \right\}_{n = 1}^{\infty}
        $ - числовые последовательности.
        Пусть $
        \exists
        \lim_{n \to \infty} a_n =
        \lim_{n \to \infty} b_n =
        c \in \R
        $.
        Пусть $
        \exists N \in \N: \forall n \geq N \hrarrow
        a_n \leq c_n \leq b_n
        $.
        Тогда $ \exists \lim_{n \to \infty} c_n = c $.
    \end{theorem_}

    \begin{proof}
        \[
        \lim_{n \to \infty} a_n = c \Lrarrow
        \forall \eps > 0 \quad \exists N_1 : \forall n \geq N \hrarrow a_n \in U_{\eps}(C)
        .\]
        \[
        \lim_{n \to \infty} b_n = c \Lrarrow
        \forall \eps > 0 \quad \exists N_2 : \forall n \geq N \hrarrow b_n \in U_{\eps}(C)
        .\]
        \[
        \forall \eps > 0 \qaud \exists N = max \{ N_1, N_2, N \} \hrarrow
        \begin{cases}
            a_n \in U_{\eps}(c) \\
            b_n \in U_{\eps}(c) \\
        \end{cases}
        \Rarrow
        \]
        \[
        c_n \in U_{\eps}(c) \quad \forall n \geq N \Rarrow
        \exists \lim_{n \to \infty} c_n = c
        .\]
    \end{proof}

    \begin{theorem}
        Пусть $ \exists \lim_{n \to \infty} x_n = +\infty $ и
        $ \exists N \in \N: \forall n \geq N \hrarrow y_n \geq x_n $.
        Тогда $ \exists \lim_{n \to \infty} y_n = +\infty $.
        Аналогично для $ -\infty $.
    \end{theorem}

    \begin{proof}
        \[
        \forall \eps > 0 \quad \exists N(\eps) \in \N:
        \quad \forall n \geq N(\eps) \hrarrow x_n > \frac{1}{\eps}.
        \] \[
        \forall \eps > 0 \quad \exists  \overline{N}(\eps) = max \{ N(\eps), N \} \in \N:
        \] \[
        \forall n \geq \overline{N}(\eps) \hrarrow y_n \geq x_n > \frac{1}{\eps} \Rarrow
        \] \[
        \lim_{n \to \infty} y_n = +\infty.
        \]
    \end{proof}

    \section{Пределы монотонных последовательностей}

    \begin{definition}
        Последовательность $ \left\{x_n \right\}_{n = 1}^{\infty} $ называется нестрого
        возрастающей(нестрого убывающей), если
        $ \forall n \in \N \hrarrow x_{n+1} \geq x_n (x_{n+1} \leq x_n) $.
    \end{definition}

    \begin{definition}
        Последовательность $ \left\{x_n \right\}_{n = 1}^{\infty} $ называется строго
        возрастающей(строго убывающей), если
        $ \forall n \in \N \hrarrow x_{n+1} > x_n (x_{n+1} < x_n) $.
    \end{definition}

    \begin{definition}
        Последовательность $ \left\{x_n \right\}_{n = 1}^{\infty} $ называется монотонной,
        если она нестрого убывающая или нестрого возрастающая.
    \end{definition}

    \begin{theorem_}[О пределе монотонной последовательности]
        Любая монотонная последовательность имеет предел в $ \oR $.

        Если $ \left\{x_n \right\}_{n = 1}^{\infty} $ нестрого возрастает, то
        $ \exists \lim_{n \to \infty} x_n = \sup \{ x_n \} $.

        Если $ \left\{x_n \right\}_{n = 1}^{\infty} $ нестрого убывает, то
        $ \exists \lim_{n \to \infty} x_n = \inf \{ x_n \} $.
    \end{theorem_}

    \begin{proof}[Докажем для нестрого возрастающей]
        \[ M = \sup{A} \Lrarrow
        \begin{cases}
            a \leq M \quad \forall a \in A \\
            \forall M' < M \quad \exists a \in A: M' < a \leq M \\
        \end{cases}
        \]
        \[
        \forall \eps > 0 \quad \exists a \in U_{\eps}(M) \cap A
        .\]
        \[
            M = \sup{x_n} \Rarrow
            \forall \eps > 0 \quad \exists N(\eps): X_{N(\eps)} \in U_{\eps}(M)
        .\]

        \begin{center}
            $  \{ x_n \} $ - возростает.
        \end{center}
        \[ \Downarrow \] \[
        x_n \geq x_{N(\eps)} \quad \forall n \geq N(\eps).
        \] \[ \Downarrow \] \[
        x_n \in U_{\eps}(M) \quad \forall n \geq N(\eps). \quad (x_n \leq M )
        \] \[ \Downarrow \] \[
        \forall \eps > 0 \quad \exists N(\eps) \in \N: \forall n \geq N(\eps) \hrarrow
        x_n \in U_{\eps}(M).
        \] \[ \Downarrow \] \[
            M = \lim_{n \to \infty} x_n.
        \]
    \end{proof}

    \section{Подпоследовательности и частичные пределы}

    \begin{definition}
        Пусть дана числовая последовательность $ \left\{x_n \right\}_{n = 1}^{\infty} $.
        Последовательность $ \left\{y_k \right\}_{k = 1}^{\infty} $ называется
        подпоследовательностью последовательности $ \left\{x_n \right\}_{n = 1}^{\infty} $,
        если существует строго возрастающая последовательность
        $ \left\{n_k \right\}_{k = 1}^{\infty} \subseteq \N:
        y_k = x_{n_k} \forall k \in \N $.
    \end{definition}

    \begin{definition}
        Будем говорить, что $ A \in \oR $ - частичный предел последовательности
        $ \left\{x_n \right\}_{n = 1}^{\infty} $, если $ \exists \{ x_{n_k} \} $ -
        подпоследовательность $ \{ x_n \} $:
        \lim_{k \to \infty} x_{n_k} = A $
    \end{definition}

    \begin{theorem_}[Критерий частичного предела]
        Пусть $ \left\{x_n \right\}_{n = 1}^{\infty} $ - числовая последовательность.
        Пусть $ A \in \oR $
        Следущие условия эквивалентны.
        \begin{enumerate}
            \item $ A $ - ч.п. $ \left\{x_n \right\}_{n = 1}^{\infty} $.
            \item $ \forall \eps > 0 $ в $ U_{\eps}(A) $ содержатся значения
                бесконечного количества элементов последовательности
                $ \left\{x_n \right\}_{n = 1}^{\infty} $.
            \item $ \forall \eps > 0 \quad \forall N \in \N \quad \exists n \geq N:
                x_n \in U_{\eps}(A) $.
        \end{enumerate}
    \end{theorem_}

    \begin{proof}[$ (1) \Rarrow (2) $]
        Пусть $ A $ - ч.п. $ \Rarrow \exists \left\{n_k \right\}_{k = 1}^{\infty} \subset \N $,
        возрастающая: $ \lim_{k \to \infty} x_{n_k} = A \Lrarrow
        \forall \eps > 0 \quad \exists K(\eps) \in \N: \forall k \geq K(\eps) \hrarrow
        x_{n_k} \in U_{\eps}(a) $.

        Т.к. $ \exists  $ бесконечно много чисел $ k \in \N: k \geq K(\eps) $, в
        $  U_{\eps}(A) $ содержатся значения бесконечного количества элементов
        $ \left\{x_n \right\}_{n = 1}^{\infty} $.
    \end{proof}

    \begin{proof}[$ (2) \Rarrow (3) $]
        Фиксируем произвольный $ \eps > 0 \Rarrow $ в $ U_{\eps}(A) $ содержатся
        значения бесконечного количества элементов последовательности
        $ \left\{x_n \right\}_{n = 1}^{\infty} $.

        Пусть $ I(\eps) $ - это те натуральные индексы, что $ x_n \in U_{\eps}(A) \quad
        \forall n \in I(\eps) $. Тогда
        \[
            \forall N \in \N \quad \exists n \in I(\eps): n \geq N
        .\]
        Но т.к. $ \eps $ было выбрано произвольно
        \[
            \forall \eps > 0 \quad \forall N \in \N: x_n \in U_{\eps}(A)
        .\]
    \end{proof}

    \begin{proof}[$ (3) \Rarrow (1) $]
        Построим подпоследовательность $ \left\{ x_{n_k} \right\}_{k = 1}^{\infty} :
        \exists \lim_{k \to \infty} x_{n_k} = A $.
        \[
            n_1 = 1
        .\]
        Поскольку при $ \eps = \frac{1}{2} $ и $ N \geq 1 + n_1 \quad \exists n \geq 1 + n_1:
        x_n \in U_{\frac{1}{2} }(A) $ если построенны $ n_1 < n_2 < \dots < n_k $
        $ x_{n_i} \in U_{\frac{1}{i}}(A) $.

        \[
            n_{k + 1} = n\left(\frac{1}{k + 1}, n_k + 1\right).
        \] \[ \Downarrow \] \[
            x_{n_{k+1}} \in U_{\frac{1}{k+1}}(A).
        \] \[ \Downarrow \] \[
            \forall k \in \N \quad \exists n_k = n\left(\frac{1}{k}, n_{k-1}\right) \in \N \quad
            x_{n_k} \in U_{\frac{1}{k}}(A).
        \] \[ \Downarrow \] \[
            \forall k \in \N \quad \exists n_k \in \N:
            \forall j \geq k \hrarrow x_{n_j} \in U_{\frac{1}{k}}(A).
        \] \[ \Downarrow \] \[
            \forall \eps > 0 \quad \exists K(\eps) = \left[ \frac{1}{\eps} \right] + 1:
            \forall k \geq K(\eps) \hrarrow x_{n_k} \in U_{\eps}(A).
        \]
    \end{proof}

    \section{Теорема Больцано - Вeйерштрасса}

    \begin{theorem_}[Теорема Больцано - Вейерштрасса]
        Пусть $ \left\{x_n \right\}_{n = 1}^{\infty} $ - ограниченная числовая последовательность.
        $ \exists $ хотя бы один конечный частичный предел
        $ \left\{x_n \right\}_{n = 1}^{\infty} $.
    \end{theorem_}

    \begin{proof}
        Поскольку $ \left\{x_n \right\}_{n = 1}^{\infty} $ - ограничена
        $ \exists M \geq 0: | x_n | \leq M $.

        Далее считаем $ M > 0 $, т.к. при $ M = 0 $ доказываемое утверждение очевидно.

        Пусть $ I^1 = \left[ -M, M \right] $.
        Выберим как $ I^2 $ половину отрезка $ I^1 $, содержащую значения бесконечного
        количества элементов $ \left\{x_n \right\}_{n = 1}^{\infty} $. Такая найдется,
        иначе в $ I^1 $ содержится конечное число значений элементов последовательности.

        Предположим мы построили последовательность $ I^1 \subset I^2 \subset \dots \subset I^k: $
        $ I^j $ содержит значения бесконечного количества элементов последовательности
        $ \forall j \leq k $.
        Разделим $ I^k $ на два конгруентных отрезка и выберем как $ I^{k+1} $ половину,
        содержащую значения бесконечного количества элементов
        $ \left\{x_n \right\}_{n = 1}^{\infty} $. (Такая найдется по указаным выше причинам.)

        В итоге получим бесконечную последовательность вложенных отрезков
        $ I^1 \subset I^2 \subset \dots \subset I^k $, которая еще и стягивающаяся, т.к.
        $ \left| I^k \right| = \frac{|I^{k - 1}|}{2^{k-1}} $. Из этого
        $ \exists x^* = \cap_{k=1}^\infty I^k $.

        Покажем, что $ \forall \eps > 0 $ в $ U_{\eps}(x^*) $ содержатся значения
        бесконечного количества элементов последовательности
        $ \left\{x_n \right\}_{n = 1}^{\infty} $,
        чтобы доказать, что
        $ x^* $ - ч.п..

        Действительно, из определения предела и из того, что
        $ x^* \in I^k \quad \forall k \in \N \Rarrow
        \exists K(\eps): x^* \in I^{K(\eps)} \subset U_{\eps}(x^*) \Rarrow $
        по построению получаем, что в $ I^{k(\eps)} $ содержится бесконечное количество
        значений элементов последовательности $ \left\{x_n \right\}_{n = 1}^{\infty} $.
    \end{proof}
\end{document}

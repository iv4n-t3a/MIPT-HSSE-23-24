\documentclass[a5paper, 10pt]{article}

\usepackage[english, russian]{babel}
\usepackage[T2A]{fontenc}
\usepackage[utf8]{inputenc}
\usepackage{amsmath, amsfonts, amssymb, amsthm, mathtools}
\usepackage{indentfirst}
\usepackage{soulutf8}
\usepackage{geometry}
\usepackage{ulem}
\usepackage{color}
\usepackage{hyperref}

\theoremstyle{plain}

\newtheorem*{theorem}{Th}
\newtheorem{theorem_}{Th}
\newtheorem*{statement}{St}
\newtheorem{statement_}{St}
\newtheorem{definition}{Def}
\newtheorem*{definition_}{Def}
\newtheorem{exersise}{Ex}
\newtheorem{corollary}{Cl}[theorem]
\newtheorem{corollary_}{Cl}[theorem_]

\geometry{top=20mm}
\geometry{bottom=20mm}
\geometry{left=10mm}
\geometry{right=10mm}

\newcommand{\N}{\mathbb N}
\newcommand{\Z}{\mathbb Z}
\newcommand{\Q}{\mathbb Q}
\newcommand{\R}{\mathbb R}
\newcommand{\eps}{\varepsilon}
\renewcommand{\phi}{\varphi}
\renewcommand{\kappa}{\varkappa}

\newcommand{\or}{\overline{\mathbb R}}
\newcommand{\hr}{\hat{\mathbb R}}

\newcommand{\larrow}{\leftarrow}
\newcommand{\rarrow}{\rightarrow}
\newcommand{\lrarrow}{\leftrightarrow}
\newcommand{\hrarrow}{\hookrightarrow}
\newcommand{\Larrow}{\Leftarrow}
\newcommand{\Rarrow}{\Rightarrow}
\newcommand{\Lrarrow}{\Leftrightarrow}

\hypersetup{
	linktocpage,
    colorlinks=true,
    linktoc=all,
    linkcolor=blue,
}


\begin{document}
	\pagenumbering{gobble}
	\author{Тюленев Александр Иванович\\(Конспектировал Иван-Чай)}
	\date{22.09.23}
	\title{Введение в математический анализ}

	\linespread{1.4}
	\selectfont

	\maketitle
	\newpage

	\tableofcontents

    \section{Топология числовой прямой}

    \begin{definition}
        Пусть $ E $ - непустое множество, тогда $ x $ называется точкой прикосновения
        $ E $, если $ \forall \eps > 0 \quad U_{\eps}(x) \cap E \neq \varnothing $.
    \end{definition}

    \begin{definition}
        Замыканием множества $ E $ называется множество всех точек прикосновения $ E $
        и обозначается $ cl E $.
    \end{definition}

    \begin{definition}
        Множество называется замкнутым, если оно совпадает со своим замыканием.
    \end{definition}

    \begin{remark}
        $ E \subset cl E $.
    \end{remark}

    \begin{remark}
        По опеределнию $ \varnothing $ и $ \R $ считаются замкнутыми.
    \end{remark}

    Пример $ a < b $, тогда $ \left[ a, b \right]  $ - замкнутое множество

    \begin{proof}
        Покажем, что $ \forall c \not \in [a, b] $ не является точкой прикосновения.

        \[
            \eps* =min \left\{ \frac{ | a - b |}{2}, \frac{ | a - c |}{2} \right\}
        .\]
    \end{proof}

    \begin{definition}
        Пусть $ G \subset \R $ - множестов. Будем оворить, что x - внутреняя точка G,
        если $ \exists \eps > 0: U_{\eps}(x) \subset G $
    \end{definition}

    \begin{definition}
        Внутренностью мнжества G называется множество всех его внутрених точек и обозначается
        $ int G $
    \end{definition}

    \begin{definition}
        Множество $ G_1 \subset \R $ называется открытым, если оно совподает со
        своей внутренностью.
    \end{definition}

    $ \varnothing $ и $ \R $ - открыты по определнию.

    $ int G \subset G $

    Пример открытого множества:
    $ \left( a, b \right) $ - открытое множество, $ a < b $

    \begin{proof}
        \[
        \eps =min \left\{ \frac{\left| x - a \right| }{2}, \frac{\left| b - x \right| }{2} \right\}
        \]
    \end{proof}

    Полуинтервал $ (a, b] $ не является ни открытым ни замкнутым множеством.

    \begin{proof}
        \[
            a \not \in int (a, b]
        \] \[
            b \in int (a, b]
        .\]
    \end{proof}

    Пример:
    $ cl \Q = \R(1) $,
    $ int \Q = \varnothing(2) $.

    \begin{proof}[(1)]
        В любом интервале найдется рациональная точка.
    \end{proof}

    \begin{proof}[(2)]
        $ \forall (a, b) $ найдется иррациональная точка.
    \end{proof}

    \begin{proof}[Докажем, что любой интервал содержит рациональную точку]
        \textbf{Это доказательство пока в разработке}
    \end{proof}

    \begin{proof}[Докажем, что любой интервал содержит иррациональную точку]
        Через гомотетирование открезка на $ \left[ 0, 2 \right] $, содержащего
        $ \sqrt{2} $
    \end{proof}

    \begin{definition}
        $ x \in \R $ называется изолированой точкой множества $ E $,
        если $ \exists \eps > 0: U_{\eps}(x) \cap E = \left\{ x \right\} $.
    \end{definition}

    \begin{definition}
        $ x \in \R $ называется предельной точкой множества
        $ E $, елси $ \forall \eps > 0 \hrarrow (U_{\eps}(x) \setminus \{x\}) \cap E
        \neq \varnothing $
    \end{definition}

    \begin{statement}
        $ x $ - точка прикосновения $ \Lrarrow  $
            $ x $ - изолированая точка $ \lor $
            $ x $ - предельная точка
        \end{aligned}
        \right.
    \end{statement}

    \begin{proof}[$ \Larrow $]
        Очевидно
    \end{proof}

    \begin{proof}[$ \Rarrow $]
        Пусть x - точка прикосновения $ \Rarrow \forall \eps > 0 \quad U_{\eps}(x) \cap E \neq
        \varnothing $,
        либо $ \forall \eps > 0 \quad U_{\eps}(x) \cap E $ содержит не только \{x\},
        но тогда она предельная.
    \end{proof}

    \begin{theorem}[Критерий точки прикосновения]
        Пусть $ E \neq \varnothing $ - множество. Точка x является точкой прикосновения E
        $ \Lrarrow \exists \left\{x_n \right\}_{n = 1}^{\infty} \subset E:
        \lim_{n \to \infty} x_n = x $.
    \end{theorem}

    \begin{statement}
        Пусть $ \left\{x_n \right\}_{n = 1}^{\infty} $ -числовая последовательность, тогда
        $ cl \left\{ x_n \right\} = \left\{ x_n \right\} \cup PL(\left\{ x_n \right\}) $.
    \end{statement}

    \begin{proof}
        Пусть $ \exists \left\{x_n \right\}_{n = 1}^{\infty} \subset E:
        \lim_{n \to \infty} x_n = x \Rarrow  \forall \eps > 0 \quad \exists
        N: \forall n \geq N \hrarrow x_n \in U_{\eps}(x) \Rarrow
        \forall \eps > 0 \quad \exists x = x_N \in U_{\eps}(x) \cap E $

        Пусть обратно, x - т. прикосновение множества E.
        $ \forall k \in \N $ по определению $ U_{\eps}(x) \cup E \neq \varnothing,
        \eps = \frac{1}{k} $
    \end{proof}

    \begin{definition}
        Множество $ K \subset \R $ называется компактным, елси из $ \forall $ последовательность
        точек $ \left\{x_n \right\}_{n = 1}^{\infty} \subset K $ можно выделить сходящуюся
        подпоследовательность, т.ч. $ \exists \lim_{k \to \infty} x_{n_k} = x \in K $.
    \end{definition}

    \begin{theorem}[Критерий компактности]
        Множество $ K \subset \R $ - компактно $ \Lrarrow $ оно ограничено и замкнуто.
    \end{theorem}

    \begin{proof}[ $ \Larrow $ ]
        Возьмем произвольную последовательность $ \left\{x_n \right\}_{n = 1}^{\infty} $,
        покажем, что $ \exists  $ сходящаяся в K подпоследовательнось.

        Т.к. $ \left\{x_n \right\}_{n = 1}^{\infty} $ - ограничена, то по т. Б-В
        $ \exists $ подоследовательность, которая сходится куда-то

        Пусть $ x^* = \lim_{n \to \infty} x_n $, но тогда $ x^* $
        - частичный предел $ R $, а
        $ k $ - замкнуто $ \Rarrow x^* \in K $
    \end{proof}

    \begin{proof}[ $ \Rarrow $]
        Предположим, что K - неограничено $ \Rarrow   \quad \forall j \in N \quad
        \exists x_j \in K: |x_j| > j \Rarrow \exists \left\{ x_j \right\} \subset K
        \lim_{j \to \infty} \left| x_j \right| = +\infty $ - противоречие
        $ \Rarrow  $ x = ограничена
    \end{proof}

    \begin{definition}
        Система множеств $ \left\{ U_\alpha \right\}, \alpha \in I $ называется
        покрытием множества $ E $, если $ E \subset \cup_{\alpha \in I} U_{\alpha} $
    \end{definition}

    \begin{definition}
        Система $ \left\{ U_\alpha \right\}_{\alpha \in I}  $ называется подпокрытием, если
        $ J \subset I $ и $ E \subset \cup_{p \in J} U_\beta $
    \end{definition}

    \begin{lemma}
        Из $ \forall  $ открытого покрытия K можно выделиь конечное подпокрытие.
    \end{lemma}

    \begin{exersise}
        Доказать, что если из $ \forall  $ любого открытого множества можно выделить конечное
        подпокрытие, то это множество компактно.
    \end{exersise}
\end{document}
